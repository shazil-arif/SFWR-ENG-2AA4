\documentclass[12pt]{article}

\usepackage{graphicx}
\usepackage{paralist}
\usepackage{listings}
\usepackage{booktabs}
\usepackage{amssymb}
\usepackage{hyperref}
\oddsidemargin 0mm
\evensidemargin 0mm
\textwidth 160mm
\textheight 200mm

\pagestyle {plain}
\pagenumbering{arabic}

\newcounter{stepnum}

\title{Assignment 2 Solution}
\author{Shazil Arif}
\date{\today}

\begin {document}

\maketitle

This report discusses the testing phase for the modules Set, ReactionT, CompoundT and MoleculeT. It also discusses
the results of running the same tests on the partner files. The assignment specifications
are then critiqued and various discussion questions are answered

\section{Testing of the Original Program}


Instead of writing one test case, executing it and then writing more and repeating the process, I took a different approach to testing. I wrote all my test cases for every module, covering all execution paths and then running the tests all together. This approach helped identify bugs not only within a certain method but also any issues with the integration of several methods/modules.


\section{Results of Testing Partner's Code}

My partner's code passed all of my test cases. In the first assignment, my partner's code failed for several test cases. Due to the nature of the first assignment and the design assumptions and choices that had to be made, failed test cases were expected. For this assignment there were no assumptions that had to be made by the developer as everything was made concise in the specification and thus it was expected that majority of test cases should pass. Furthermore, I did not find any problems with mine or my partners code since all tests passed

\section{Critique of Given Design Specification}

I genuinely really liked the formal specification given for this assignment. The main reason I enjoyed it was the mathematical and formal notation used to communicate clearly any requirements, an invariant for a module, boolean conditions for exceptions etc. The inputs, outputs and desired behavior was very clear. However, due to the bulk of formal notation and less natural language it took significantly longer to fully interpret and understand the specification and specifically how each module depends and integrates with other modules. For example, it took me time to understand that certain modules were interfaces with generic/abstract methods that would have their unique implementation in a module that would inherit those properties. I did find some aspects of the design that could be changed. I would make the design more essential by removing certain redundant features. For example, the MolecSet and ElmSet modules are redundant since Python is dynamically typed and does not enforce the programmer to have the same types within a list. Simply using an instance of a Set object (Set refers to Set module defined in the context of this assignment and not a generic python set) of ElementT types is no different than using an ElmSet. The requirements are actually not opaque and do not satisfy information hiding. It is actually not stated anywhere whether state variables should be private or public. The state variables should be enforced to be kept as private making the specification opaque. I did not find any examples where the specification was did not meet other quality criteria including high cohesion, low coupling and consistency

\section{Answers}

\begin{enumerate}[a)]

\item Comparing the specification of the first assignment with this one, the major difference is the how the specification is communicated. Assignment 1 was described in natural language and Assignment 2 had a formal specification that used mathematical notation and symbols to communicate specifications, invariant and certain properties of methods. 

\begin{enumerate}
\item Advantages of Natural language specification:
	\begin{enumerate}
	
	\item General audience, easy to interpret natural language vs technical terminology and notation(formal specification)
	\item Statements can be vague and not very concise regarding requirements, this allows the developer/programmer to make design choices and assumptions as necessary which encourages creativity and unique solutions
	
	\end{enumerate}
\item Disadvantages of Natural language specification:

	\begin{enumerate}
	
	\item The vagueness of natural language description may encourage creativity and unique solutions but often opens the door for confusion and may result in a product that does not meet the requirements due to lack of understanding and communication.
	\item Often difficult to communicate technical ideas verbally, it can also be more challenging to interpret by the developer. For example ``Function x takes an input y such that y is an integer less than or equal to 5'' can be harder to interpret than ``Function x takes a input y such that $y : \mathbb{Z} \leq x$''
	
	\end{enumerate}
	
	\item Advantages of a Formal specification
	
	\begin{enumerate}
		\item The requirements are clear and concise. Mathematical notation and symbols indicate exactly what is required. For example, a method's output can be described as the output of a mathematical function. Let's say a method computes lengths of sides of a triangle using the Pythagorean  theorem and so the method's specification can be be described as ``given $a$ and $b$ as parameters return $c$ = $\sqrt{a^2 + b^2}$''. This formal notation can be extended to explain certain properties or an invariant, for example ``Method x(y) where y is a parameter containing a list of integers, should raise an error $\forall a \in y. a \geq 20$''. The formal notation makes the requirements clear and avoids confusions among programmers leading to a higher quality end product.
		
	\end{enumerate}
	
	\item Disadvantages of a formal specification
	
	\begin{enumerate}
		\item Due to the technical nature of the notation, it can be difficult for certain developers to read and interpret. It requires extra effort and a basic background knowledge and familiarity with the notation and language.
	\end{enumerate}

\end{enumerate} 

\item The process of converting string to logical and meaningful syntactical components is called \emph{\href{https://en.wikipedia.org/wiki/Lexical_analysis}{Lexical Analysis}}. To convert a string to a reaction in the context of this assignment and the ReactionT module, we need to supply ReactionT a left and right hand side of a chemical equation. Suppose the user enters a equation such as ``2H2 + O2 -> 2H2O'' as a string. To convert this to a reaction, we can split the string first by the ``->'' character and then inspect individual characters such as O and map them to ElementT types. We also need to take the numbers associated with the elements (the coefficients and subscripts) and construct MoleculeT objects, then a MolecSet, CompoundT and eventually the two sets of CompoundT objects to create a ReactionT object. The main secrets are substring search and/or regex splitting which can be implemented in a new Module that could be called StringReactionT.
  
\item I would create a new module that would contain a mapping between elements and a their atomic mass. Additionally, the element would be of type ElementT and mapped to a floating point value. This may be achieved using a dictionary that would look something like :
\begin{verbatim}
	weights = { ElementT.H  : 1.007,
				ElementT.He : 4.002
			   }
\end{verbatim}

This mapping can be easily used to retrieve the weight of ElementT type. Additionally, the constit\textunderscore elements() method defined in both CompoundT and MoleculeT retrieves a list of elements contained in the compound/molecule which can be passed to a new method that will iteratively get the weights all of the elements, add them up and get the total weight of a molecule or compound. This is elegant because it does not require us to change any existing implementation and only add new code.

\item The usual convention in Chemistry is to have a chemical equation balanced with integer coefficients. The domain of Integer Programming addresses this problem. An algorithm that can compute coefficients as integers can be found \emph{\href{https://www.sciencedirect.com/science/article/pii/S0895717706000367}{here}}

\item In a dynamically typed language such as Python the types of variables are checked during execution, as the program runs. This means a programmer could write a statement such as \begin{verbatim}
print("hello" + 5)
\end{verbatim} and Python would not indicate any errors and would execute. But, as the program executes Python will throw an error along the lines of not being able to convert a string to an integer since the types are clearly mismatched. Furthermore, if the above print statement was placed inside a if statement whose condition is never true, then Python will never throw an error.\\

In a statically typed language like Java the types of variables are checked before run-time. If a programmer writes a statement such as \begin{verbatim}
System.out.println("hello" + 5)
\end{verbatim} Java will not even compile due to this type mismatch. If the print statement is placed inside a if statement that may never even execute, Java will still not compile due to the type mismatch.\\

Advantages of static typing are enforcing a coding standard. The strict type checking makes it harder to get past the compiler and as a result when the program executes, gives the programmer high confidence that they have done things correctly as opposed to a dynamically typed language where this confidence may be lower since some type errors may never even be caught. Static typing also results in better performance at run time since the type of every variable does not need to be checked.\\

Disadvantages of static typing include frustration, constraints on a program and making simple programs unnecessarily complicated due to type checking. Some programmers may argue that constant errors showing up can be frustrating during the development process and leads to a decrease in productivity. Static typing places constraints on a program since it can only worth with a given type at a time. For example, a programmer
may want to write a method to convert numbers to string and support this for both integers and floating point values. Unfortunately, due to static type checking the method can only take one type of value and do the conversion and a additional method must be defined to do the same process but for a different type of the input. This leads to the last point; making simple programs complicated. Consider the Quicksort sorting algorithm. We want to implement a generic quick sort that can be used on any comparable type such as strings and integers. In a dynamically typed language this is not a problem, we can have a function sort() that can take a list of any type. However, it is more complicated in a statically typed language such as Java. This is exactly why Java has \emph{\href{https://www.geeksforgeeks.org/generics-in-java/}{Generics}}. In conclusion, the strict type checking leads to more and usually complicated code as opposed to dynamic typing.

\item \begin{verbatim} [(x,y) for x in range(1,11) for y in range(x,11) if x % 2 and y % 2 and x < y] 
\end{verbatim}

\item \begin{verbatim}
def length(x):
        return sum(map(lambda n:1,x))
\end{verbatim}

\item The interface can be thought of as a ``contract'' between a client and a programmer that may state the public methods in a class, their parameters and return values. The interface communicates to a client the necessary knowledge to use the methods in a module without having to worry about how the module works internally. The implementation is the underlying code, algorithms and secrets used to achieve the output stated in the interface of the module. For example, the interface of a math module may define: \begin{verbatim}
def sin(x)
\end{verbatim} as a method that returns the sine of a value x in radians. A client can easily call this method and get the correct output values, they do not have to worry about how the value is being computed. In contrast, the implementation is the underlying algorithm/secret that is used to compute the sine of a value which is hidden from the client. The method may use a Taylor series approximation to compute sin(x) but a client does not to worry about this.

\item The following Software Engineering principles should guide the design of a module's interface as follows:
	\begin{enumerate}
	
		\item Abstraction
			\begin{enumerate}
					\item Abstraction is the idea of ``abstracting'' unnecessary details of a software module interface from a client. The programmer should only provide the details necessary for a client to achieve their desired output. Abstraction should encourage the idea of focusing on the \emph{interface} rather than \emph{implementation}. This means state information should be encapsulated, the interface should not expose any details about internal algorithms/secrets (potentially from a security perspective for a sensitive application such as a password hashing module, but more importantly to hide these details from the client to avoid confusion).
			\end{enumerate}
			
		\item Anticipation of Change
			\begin{enumerate}
					\item Anticipation of change (AoC) addresses software design from a \emph{long term and maintainable} perspective. Suppose we design a module that satisfied our clients needs but six months later, we need to implement new functionality/features. By AoC principle to implement these features we should be able to do so smoothly and elegantly. This means not a lot of existing code should be modified but instead new code should be able to be easily added and integrated. This encourages the design of software in \emph{modules} and each module should address a different concern/sub problem and the module as a whole should have high cohesion. This also means that software should have low coupling since high coupling would imply changing one component of the code would not work without changing another component.
			\end{enumerate}
		
		\item Generality
			\begin{enumerate}
			
			\item
					
			\end{enumerate}
			
		\item Modularity
			\begin{enumerate}
					\item Modularity ties in with Anticipation of Change (AoC). It can be viewed as a consequence of AoC. As we discussed above AoC encourages software to be designed in module/smaller units to increase cohesion and achieve long term maintainability. This means we should have seperate classes that group together various methods used to achieve a common task. For example, in a Chemical equations software we should have a module that addresses balancing of equations and a seperate module that addresses the representation of an equation. These two modules would then have their own methods that achieve the sub problem that they are addressing (a balancing method in the balancing module)
			\end{enumerate}
			
		\item Separation of concerns
			\begin{enumerate}
					\item Seperation of conerns (SoC) specifies the need for humans to work with little information at any given time. Humans have a  difficult time understanding all the complex and technical details to design a software system. To avoid confusion and enable a person to be able to work on a software, we should work with small bits of information/work with smaller sub problems at a time. This encourges the design of software to be done in modules (enabling Modularity)
			\end{enumerate}
	\end{enumerate}
  
\end{enumerate}

\newpage

\lstset{language=Python, basicstyle=\tiny, breaklines=true, showspaces=false,
  showstringspaces=false, breakatwhitespace=true}
%\lstset{language=C,linewidth=.94\textwidth,xleftmargin=1.1cm}

\def\thesection{\Alph{section}}

\section{Code for ChemTypes.py}

\noindent \lstinputlisting{../src/ChemTypes.py}

\newpage

\section{Code for ChemEntity.py}

\noindent \lstinputlisting{../src/ChemEntity.py}

\newpage

\section{Code for Equality.py}

\noindent \lstinputlisting{../src/Equality.py}

\newpage

\section{Code for Set.py}

\noindent \lstinputlisting{../src/Set.py}

\newpage

\section{Code for ElmSet.py}

\noindent \lstinputlisting{../src/ElmSet.py}

\newpage

\section{Code for MolecSet.py}

\noindent \lstinputlisting{../src/MolecSet.py}

\newpage

\section{Code for CompoundT.py}

\noindent \lstinputlisting{../src/CompoundT.py}

\newpage

\section{Code for ReactionT.py}

\noindent \lstinputlisting{../src/ReactionT.py}

\newpage

\section{Code for test\_All.py}

\noindent \lstinputlisting{../src/test_All.py}

\newpage

\section{Code for Partner's Set.py}

\noindent \lstinputlisting{../partner/Set.py}

\newpage

\section{Code for Partner's MoleculeT.py}

\noindent \lstinputlisting{../partner/MoleculeT.py}

\newpage

\section{Code for Partner's CompoundT.py}

\noindent \lstinputlisting{../partner/CompoundT.py}

\newpage

\section{Code for Partner's ReactionT.py}

\noindent \lstinputlisting{../partner/ReactionT.py}

\end {document}
