\documentclass[12pt]{article}

\usepackage{graphicx}
\usepackage{paralist}
\usepackage{amsfonts}
\usepackage{amsmath}
\usepackage{hhline}
\usepackage{booktabs}
\usepackage{multirow}
\usepackage{multicol}
\usepackage{listings}
\usepackage{xcolor}
\usepackage{url}
\lstset{
  basicstyle=\ttfamily,
  mathescape
}

\oddsidemargin -10mm
\evensidemargin -10mm
\textwidth 160mm
\textheight 200mm
\renewcommand\baselinestretch{1.0}

\pagestyle {plain}
\pagenumbering{arabic}

\newcounter{stepnum}

%% Comments

\usepackage{color}

\newif\ifcomments\commentstrue

\ifcomments
\newcommand{\authornote}[3]{\textcolor{#1}{[#3 ---#2]}}
\newcommand{\todo}[1]{\textcolor{red}{[TODO: #1]}}
\else
\newcommand{\authornote}[3]{}
\newcommand{\todo}[1]{}
\fi

\newcommand{\wss}[1]{\authornote{blue}{SS}{#1}}

\title{Assignment 3, Part 1, Specification}
\author{SFWR ENG 2AA4}

\begin {document}

\maketitle
This Module Interface Specification (MIS) document contains modules, types and
methods for implementing a generic 2D sequence that is instantiated for both
land use planning and for a Discrete Elevation Model (DEM).

In applying the specification, there may be cases that involve undefinedness.
We will interpret undefinedness following~\cite{Farmer2004}:

If $p: \alpha_1 \times .... \times \alpha_n \rightarrow \mathbb{B}$ and any of
$a_1, ..., a_n$ is undefined, then $p(a_1, ..., a_n)$ is False.  For instance,
if $p(x) = 1/x < 1$, then $p(0) = \text{False}$.  In the language of our
specification, if evaluating an expression generates an exception, then the
value of the expression is undefined.

\wss{The parts that you need to fill in are marked by comments, like this one.
  In several of the modules local functions are specified.  You can use these
  local functions to complete the missing specifications.}

\wss{As you edit the tex source, please leave the \texttt{wss} comments in the
  file.  Put your answer \textbf{after} the comment.  This will make grading
  easier.}

\bibliographystyle{plain}
\bibliography{SmithCollectedRefs}

\newpage

\section* {Land Use Type Module}

\subsection*{Module}

LanduseT

\subsection* {Uses}

N/A

\subsection* {Syntax}

\subsubsection* {Exported Constants}

None

\subsubsection* {Exported Types}

Landtypes = \{R, T, A, C\}\\

\noindent \textit{//R stands for Recreational, T for Transport, A for Agricultural, C for
  Commercial}

\subsubsection* {Exported Access Programs}

\begin{tabular}{| l | l | l | p{5cm} |}
\hline
\textbf{Routine name} & \textbf{In} & \textbf{Out} & \textbf{Exceptions}\\
\hline
new LanduseT & Landtypes & LanduseT & ~\\
\hline
\end{tabular}

\subsection* {Semantics}

\subsubsection* {State Variables}

landuse: Landtypes

\subsubsection* {State Invariant}

None

\subsubsection* {Access Routine Semantics}

\noindent new LandUseT($t$):
\begin{itemize}
\item transition: $\mathit{landuse} := t$
\item output: $out := \mbox{self}$
\item exception: none
\end{itemize}

\subsubsection* {Considerations}

When implementing in Java, use enums (as shown in Tutorial 06 for ElementT).

\newpage

\section* {Point ADT Module}

\subsection*{Template Module inherits Equality(PointT)}

PointT

\subsection* {Uses}

N/A

\subsection* {Syntax}

\subsubsection* {Exported Types}

\textcolor{blue}{PointT = ?}

\subsubsection* {Exported Access Programs}

\begin{tabular}{| l | l | l | l |}
\hline
\textbf{Routine name} & \textbf{In} & \textbf{Out} & \textbf{Exceptions}\\
\hline
PointT & $\mathbb{Z}$, $\mathbb{Z}$ & PointT & \\
\hline
row & ~ & $\mathbb{Z}$ & ~\\
\hline
col & ~ & $\mathbb{Z}$ & ~\\
\hline
translate & $\mathbb{Z}$, $\mathbb{Z}$ & PointT & ~\\
\hline
\end{tabular}

\subsection* {Semantics}

\subsubsection* {State Variables}

$r$: \textcolor{blue}{$\mathbb{Z}$}\\
$c$: \textcolor{blue}{$\mathbb{Z}$}\\

\subsubsection* {State Invariant}

None

\subsubsection* {Assumptions}

The constructor PointT is called for each object instance before any other
access routine is called for that object.  The constructor cannot be called on
an existing object.

\subsubsection* {Access Routine Semantics}

PointT($row, col$):
\begin{itemize}
\item transition: \textcolor{blue}{$r := row, c := col$}

\item output: $out := \mathit{self}$
\item exception: None
\end{itemize}

\noindent row():
\begin{itemize}
\item output: $out := r$
\item exception: None
\end{itemize}

\noindent col():
\begin{itemize}
\item \textcolor{blue}{output: $out := c$}
\item exception: None
\end{itemize}

\noindent translate($\Delta r$, $\Delta c$):
\begin{itemize}
\item \textcolor{blue}{output: $out :=$ PointT($r + \Delta r$, $c + \Delta c$)}
\item exception: \textcolor{blue}{None}
\end{itemize}

\newpage

\section* {Generic Seq2D Module}

\subsection* {Generic Template Module}

Seq2D(T)

\subsection* {Uses}

PointT

\subsection* {Syntax}

\subsubsection* {Exported Types}

Seq2D(T) = ?

\subsubsection* {Exported Constants}

None

\subsubsection* {Exported Access Programs}

\begin{tabular}{| l | l | l | p{6cm} |}
\hline
\textbf{Routine name} & \textbf{In} & \textbf{Out} & \textbf{Exceptions}\\
\hline
Seq2D & seq of (seq of T), $\mathbb{R}$ & Seq2D & IllegalArgumentException\\
\hline
set & PointT, T & ~ & IndexOutOfBoundsException\\
\hline
get & PointT & T & IndexOutOfBoundsException\\
\hline
getNumRow & ~ & $\mathbb{N}$ & \\
\hline
getNumCol & ~ & $\mathbb{N}$ & \\
\hline
getScale & ~ & $\mathbb{R}$ & \\
\hline
count & T & $\mathbb{N}$ & \\
\hline
countRow & T, $\mathbb{N}$ & $\mathbb{N}$ & \\
\hline
area & T & $\mathbb{R}$ & \\
\hline
\end{tabular}

\subsection* {Semantics}

\subsubsection* {State Variables}

$s$: seq of (seq of T)\\
scale: $\mathbb{R}$\\
nRow: $\mathbb{N}$\\
nCol: $\mathbb{N}$

\subsubsection* {State Invariant}

None

\subsubsection* {Assumptions}

\begin{itemize}
\item The Seq2D(T) constructor is called for each object instance before any
other access routine is called for that object.  The constructor can only be
called once.
\item Assume that the input to the constructor is a sequence of rows, where each
  row is a sequence of elements of type T.  The number of columns (number of
  elements) in each row is assumed to be equal. That is each row
  of the grid has the same number of entries.  $s[i][j]$ means the ith row and
  the jth column.  The 0th row is at the top of the grid and the 0th column
  is at the leftmost side of the grid.
\end{itemize}

\subsubsection* {Access Routine Semantics}

Seq2D($S$, scl):
\begin{itemize}
\item \textcolor{blue}{transition (note that the list does not enforce an \emph{order} in which the transitions occur, only the transitions that must occur): \begin{enumerate}\item $s := S$  \item $scale := scl$ \item $nRow := |S|$ \item $nCol := |S[0]|$\end{enumerate}}

\item output: $\mathit{out} := \mathit{self}$
\item \textcolor{blue}{exception: \\
exc $:=$ (scl $<$ 0) $\lor$ (S = $<>$) $\lor$ ($| S[0] | = 0$) $\lor$ ($\exists (x : $ Seq of T $| x \in S[ 1..|S|-1 ] : |x| \neq | S[0] |$)) $\implies$ IllegalArgumentException}
\end{itemize}

\noindent set($p, v$):
\begin{itemize}
\item transition: \textcolor{blue}{$s[p.row()][p.col()] = v$}
\item exception:\\ \textcolor{blue}{exc $:=$ (p.row() $ >= $ nRow) $\lor$ (p.col() $ >= $ nCol) $\lor$ (p.row() $ < $ 0) $\lor$ (p.col() $ < $ 0)$\implies$ IndexOutOfBoundsException}
\end{itemize}

\noindent get($p$):
\begin{itemize}
\item output: \textcolor{blue}{$out := s[p.row()][p.col()]$}
\item exception:\\ \textcolor{blue}{exc $:=$ (p.row() $ >= $ nRow) $\lor$ (p.col() $ >= $ nCol) $\lor$ (p.row() $ < $ 0) $\lor$ (p.col() $ < $ 0) $\implies$ IndexOutOfBoundsException}
\end{itemize}

\noindent getNumRow():
\begin{itemize}
\item output: $out := \mbox{nRow}$
\item exception: None
\end{itemize}

\noindent getNumCol():
\begin{itemize}
\item output: $out := \mbox{nCol}$
\item exception: None
\end{itemize}

\noindent getScale():
\begin{itemize}
\item output: $out := \mbox{scale}$
\item exception: None
\end{itemize}

\noindent count($t$: T):
\begin{itemize}
\item \textcolor{blue}{output: out $:= (+i : \mathbb{N} | i \in [0..|s|-1] : $ countRow($t$,$i$))}
\item exception: None
\end{itemize}

\noindent countRow($t$: T, $i: \mathbb{N}$):
\begin{itemize}
\item \textcolor{blue}{output: out $:= (+x : $ T $| x \in s[i] \land x = t : 1)$}
\item \textcolor{blue}{exception: exc $ := \lnot$ validRow($i$) $\implies$ IndexOutOfBoundsException}
\end{itemize}

\noindent area($t$: T):
\begin{itemize}
\item \textcolor{blue}{output: $out :=$ count($t$) $ * (scale * scale)$}
\item exception: None
\end{itemize}

\subsection*{Local Functions}

\noindent validRow: $\mathbb{N} \rightarrow \mathbb{B}$\\
\noindent \textcolor{blue}{validRow($r$) $\equiv r \geq 0 \land (r < $ nRow)} \\

\noindent validCol: $\mathbb{N} \rightarrow \mathbb{B}$\\
\noindent \textcolor{blue}{validCol($c$) $\equiv (c \geq 0) \land (c < $ nCol)} \\


\noindent validPoint: $\mbox{PointT} \rightarrow \mathbb{B}$\\
\noindent \textcolor{blue}{validPoint($p$) $\equiv$ validCol(p.col()) $\land$ validRow(p.row())}


\newpage

\section* {LanduseMap Module}

\subsection* {Template Module}

\noindent \textcolor{blue}{LanduseMap is Seq2D(LanduseT)}


\newpage

\section* {DEM Module}

\subsection* {Template Module}

DemT is Seq2D($\mathbb{Z}$)

\subsection* {Syntax}

\subsubsection* {Exported Access Programs}

\begin{tabular}{| l | l | l | p{6cm} |}
\hline
\textbf{Routine name} & \textbf{In} & \textbf{Out} & \textbf{Exceptions}\\
\hline
total & & $\mathbb{Z}$ & \\
\hline
max &  & $\mathbb{Z}$ & \\
\hline
ascendingRows & & $\mathbb{B}$ & \\
\hline
\end{tabular}

\subsection* {Semantics}

\subsubsection* {Access Routine Semantics}

\noindent total(): 
\begin{itemize}
\item \textcolor{blue}{output : out $:= +(x, y : \mathbb{N} |$ validRow($x$) $\land$ validCol($y$) : $s[x][y]$)}
\item exception: None
\end{itemize}

\noindent max():
\begin{itemize}
\item \textcolor{blue}{output: out $:=$ $M$ such that $\forall (x : $ Seq of $\mathbb{Z}$ $| x \in s : \forall (y : \mathbb{Z} | y \in x : M \geq y )) $}
\item exception: None
\end{itemize}

\noindent ascendingRows():
\begin{itemize}
\item \textcolor{blue}{output:out $:= \forall (i : \mathbb{N} | i \in [0..|s|-2] : sum(s[i]) < sum(s[i+1])) $}
\item exception: None
\end{itemize}

\subsection*{Local Functions}

\noindent validRow: $\mathbb{N} \rightarrow \mathbb{B}$\\
\noindent \textcolor{blue}{validRow($n$) $\equiv$ ($n \geq 0$) $\land$ ($n < $ nRow)}\\

\noindent validCol: $\mathbb{N} \rightarrow \mathbb{B}$\\
\noindent \textcolor{blue}{validCol($c$) $\equiv$ ($c \geq 0$) $\land$ ($c < $ nCol)}\\
  
\noindent sum: Seq of $\mathbb{Z} \rightarrow \mathbb{Z}$\\
\noindent \textcolor{blue}{sum($s$) $\equiv$ $(+x : \mathbb{Z} | x \in s: x)$}



\newpage

\section*{Critique of Design}


I would change the specification to avoid stating to use an ArrayList and provide no specific implementation details. This is to free the programmer from any implementation constraints and let that be their own design decision as long as the specification is satisfied. For the repeated local functions (validRow() and validCol()) since they have the exact same implementation I would make them a publicly accessible routine in Seq2D and DemT can then inherit the method. \\

This can also potentially make Seq2D easier to use and require a client to write less code. For example, suppose the client wants to traverse the 2D sequence horizontally and diagonally then with the current implementation they can use a loop and a set of variables that should be within the bounds of the 2D sequence. Although they can get the number of rows and columns with getNumRow() and getNumCol() respectively, they have to write out the full logic for checking if their variables are within bounds which may look something like:\\
\begin{verbatim}
Assuming ``maze'' is an instance of Seq2D of some type
while(i >= 0 && i < maze.getNumRow() && j >= 0 && j < maze.getNumCol()){
			//traverse/access maze[i][j] etc.
}
\end{verbatim}

The above loop condition gets messy and complicated. But with publicly accessible methods validRow() and validCol() it can be cleanly written as:\\
\begin{verbatim}
Assuming ``maze'' is an instance of Seq2D of some type
while(validRow(i) && validCol(j)){
			//traverse/access maze[i][j] etc.
}
\end{verbatim}

Furthermore, if validRow() is publicly accessible then a client can use to verify a row number before calling countRow(). Same idea for set(), a client can validate their row and column values before passing a point object to set(), adding additional safety to the software usage and thus prone to less errors/exceptions.


\begin{enumerate}
\item The original version of the assignment had an Equality interface defined
  as for A2, but this idea was dropped.  In the original version Seq2D inherited
  the Equality interface.  Although this works in Java with the LanduseMapT, it is
  problematic for DemT.  Why is it problematic?  (Hint: DEMT is instantiated
  with the Java type Integer.)\\
  
  In Java, the type Integer is a reference type which is different from the primitive type int. Integer inherits from Object and as a result equality of two Integers is defined as the equality of their references, i.e two Integer values are the same if they point to the same memory location. For LanduseMapT it is instantiated with type LandUseT which is an enumeration. enums in Java is not an object but rather a special data type. Two enum objects are considered equal if either they reference the same memory location or their values are equal. Consider this code:
  \begin{verbatim}
enum Color {
    	Red, Green, Blue;
	}
	public class MyClass {
    	public static void main(String args[]) {
      		Integer p = new Integer(5);
      		Integer q = new Integer(5);
      		System.out.println(q.equals(p));
      		System.out.println(q == p);
      
     		 Color c1 = Color.Green;
      		Color c3 = Color.Green;
      		System.out.println(c3.equals(c1));
      		System.out.println(c3 == c1);
    }
}
\end{verbatim}

The output of the above code is :\\
true\\
false\\
true\\
true\\

Notice that for the Integer object, if the two objects have the same value but are instantiated separately (i.e they reference different memory locations) then they are not considered equal. This requires explicitly using .equals()\\

However, for the enum object even if two objects have the same value but reference different memory locations they are still equal. Thus, it is problematic to inherit a Equality interface for DemT due to definition of Equality for Integer versus enum in Java.
  
  
\item Although Java has several interfaces as part of the standard language,
  such as the Comparable interface, there is no Equality interface.  Instead
  equals is provided through inheritance from Object.  Why do you think the
  Java language designers decided to use inheritance for equality, instead of
  providing an interface?\\
  
  I believe this choice was made due to the nature of equality of some types in Java. Generally, two objects are considered equal if they reference the same memory location but certain types do not have to meet this requirement(e.g enums as discussed above). So to avoid ambiguity when it comes to what ``Equality'' is defined as, .equals() is provided through inheritance from Object since this will always compare memory references and this allows for a standard definition of ``Equality'' for all the different types/objects in Java.
  
\item The qualities of good module interface push the design of the interface in
  different directions. Why is it rarely possible to achieve a module interface
  that simultaneously is essential, minimal and general?\\
  
  A module is essential if it has no redundant features. This means there are no methods/routines, state variables and invariant that can be removed. A module is minimal if for each routine we have as few state transitions as possible and different services/transitions are independent of one another. A module is general if it is designed without a specific use in mind but rather address a broader domain. For example, consider the scipy library in python. It is not designed to allow users to do matrix operations or solve systems of differential equations. It addresses the general problem of $emph{Scientific Computing}$ and working with matrices and differential equations falls under these. What makes the scipy library general is that there is no specific use for an ODE solver. A client may use it solve a spring-mass system, an electric circuit or even calculate the position of planets but scipy does not enforce any of this, it simply requires a generic set of parameters such as initial conditions, a list of functions etc. 
  
  It is rarely possible to achieve a module interface that simultaneously meets the criteria of being essential, minimal and general.
  
  
\end{enumerate}

\end {document}