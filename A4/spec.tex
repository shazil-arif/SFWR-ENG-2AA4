\documentclass[12pt]{article}

\usepackage{graphicx}
\usepackage{paralist}
\usepackage{amsfonts}
\usepackage{amsmath}
\usepackage{hhline}
\usepackage{booktabs}
\usepackage{multirow}
\usepackage{hyperref}
\usepackage{multicol}
\usepackage{listings}
\usepackage{xcolor}
\usepackage{url}
\hypersetup{
    colorlinks=true,
    linkcolor=blue,
    filecolor=magenta,      
    urlcolor=cyan,
}
\lstset{
  basicstyle=\ttfamily,
  mathescape
}

\oddsidemargin -10mm
\evensidemargin -10mm
\textwidth 160mm
\textheight 200mm
\renewcommand\baselinestretch{1.0}

\pagestyle {plain}
\pagenumbering{arabic}

\newcounter{stepnum}

%% Comments

\usepackage{color}

\newif\ifcomments\commentstrue

\ifcomments
\newcommand{\authornote}[3]{\textcolor{#1}{[#3 ---#2]}}
\newcommand{\todo}[1]{\textcolor{red}{[TODO: #1]}}
\else
\newcommand{\authornote}[3]{}
\newcommand{\todo}[1]{}
\fi

\newcommand{\wss}[1]{\authornote{blue}{SS}{#1}}

\title{Assignment 4, Two Dots game Specification}
\author{SFWR ENG 2AA4}

\begin {document}

\maketitle
This Module Interface Specification (MIS) document contains interfaces,ADT's and
methods for implementing a game of Two Dots. Some modules and notation were borrowed from the A3 specification and all credit goes to it. The specification is generally formal but contains some informal parts for specifying for example; output to a screen, starting a timer and describing a iterative method which has no output. The strategy pattern was used in the Strategy interface and the classes that inherit it for different game modes. The singleton pattern was followed for the Timer module and MVC was used throughout to structure the interface of the application and to seperate concerns. A full MIS of the View and Controller is also provided. Comments are provided in the ``Considerations'' section at the end of each module where appropriate to clarify any misunderstandings and ambiguities.

\newpage

\section* {Color Module}

\subsection*{Module}

Color

\subsection* {Uses}

N/A

\subsection* {Syntax}

\subsubsection* {Exported Constants}

None

\subsubsection* {Exported Types}

Color = \{R, G, B, P, Y\}\\

\noindent \textit{//R stands for Red, G for green, B for blue, P for
  Purple, Y for yellow}

\subsubsection* {Exported Access Programs}

\begin{tabular}{| l | l | l | p{5cm} |}
\hline
\textbf{Routine name} & \textbf{In} & \textbf{Out} & \textbf{Exceptions}\\
\hline
randomColor &  & Color & ~\\
\hline
\end{tabular}

\subsection* {Semantics}

\subsubsection* {State Variables}

colors: Color

\subsubsection* {State Invariant}

None

\subsubsection* {Access Routine Semantics}

\noindent randomColor():
\begin{itemize}
\item transition: none
\item output: $out :=$ randomVal()
\item exception: none
\end{itemize}


\subsection*{Local Functions}

\noindent randomVal(): Color\\
\noindent randomVal() $\equiv (i = 0 \implies $R $| i = 1 \implies$ G $| i = 2 \implies$ B $| i = 3 \implies$ P $| i = 4 \implies$ Y ) 
Where $i$ is a uniformly-distributed random number in the range $0 \leq i \leq 4$ \\

\newpage

\section* {Point ADT Module}

\subsection*{Template Module}

PointT

\subsection* {Uses}

N/A

\subsection* {Syntax}

\subsubsection* {Exported Types}

\textcolor{blue}{PointT = ?}

\subsubsection* {Exported Access Programs}

\begin{tabular}{| l | l | l | l |}
\hline
\textbf{Routine name} & \textbf{In} & \textbf{Out} & \textbf{Exceptions}\\
\hline
PointT & $\mathbb{Z}$, $\mathbb{Z}$ & PointT & \\
\hline
row & ~ & $\mathbb{Z}$ & ~\\
\hline
col & ~ & $\mathbb{Z}$ & ~\\
\hline
\end{tabular}

\subsection* {Semantics}

\subsubsection* {State Variables}

$r$: \textcolor{blue}{$\mathbb{Z}$}\\
$c$: \textcolor{blue}{$\mathbb{Z}$}\\

\subsubsection* {State Invariant}

None

\subsubsection* {Assumptions}

The constructor PointT is called for each object instance before any other
access routine is called for that object.  The constructor cannot be called on
an existing object.

\subsubsection* {Access Routine Semantics}

PointT($row, col$):
\begin{itemize}
\item transition: \textcolor{blue}{$r := row, c := col$}

\item output: $out := \mathit{self}$
\item exception: None
\end{itemize}

\noindent row():
\begin{itemize}
\item output: $out := r$
\item exception: None
\end{itemize}

\noindent col():
\begin{itemize}
\item \textcolor{blue}{output: $out := c$}
\item exception: None
\end{itemize}

\newpage

\section* {Generic Board Module}

\subsection* {Generic Template Module}

Board(T)

\subsection* {Uses}

PointT

\subsection* {Syntax}

\subsubsection* {Exported Types}

Board(T) = ?

\subsubsection* {Exported Constants}

None

\subsubsection* {Exported Access Programs}

\begin{tabular}{| l | l | l | p{6cm} |}
\hline
\textbf{Routine name} & \textbf{In} & \textbf{Out} & \textbf{Exceptions}\\
\hline
Board & $\mathbb{N},\mathbb{N}$ & Board & IllegalArgumentException\\
\hline
set & PointT, T & ~ & IndexOutOfBoundsException\\
\hline
get & PointT & T & IndexOutOfBoundsException\\
\hline
getNumRow & ~ & $\mathbb{N}$ & \\
\hline
getNumCol & ~ & $\mathbb{N}$ & \\
\hline
\end{tabular}

\subsection* {Semantics}

\subsubsection* {State Variables}

$s$: seq of (seq of T)\\
nRow: $\mathbb{N}$\\
nCol: $\mathbb{N}$

\subsubsection* {State Invariant}

None

\subsubsection* {Assumptions}

\begin{itemize}
\item The Board(T) constructor is called for each object instance before any
other access routine is called for that object.  The constructor can only be
called once.
\item $s[i][j]$ means the ith row and the jth column.  The 0th row is at the top of the grid and the 0th column is at the leftmost side of the grid.
\end{itemize}

\subsubsection* {Access Routine Semantics}

Board($row$, $col$):
\begin{itemize}
\item \textcolor{blue}{transition (note that the list does not enforce an \emph{order} in which the transitions occur, only the transitions that must occur): \begin{enumerate} \item $nRow := row$ \item $nCol := col$\end{enumerate}}

\item output: $\mathit{out} := \mathit{self}$
\item \textcolor{blue}{exception: \\
exc $:=$ (row $\leq$ 0) $\lor$ (col $\leq$ 0) $\implies$ IllegalArgumentException}
\end{itemize}

\noindent set($p, v$):
\begin{itemize}
\item transition: \textcolor{blue}{$s[p.row()][p.col()] = v$}
\item exception:\\ \textcolor{blue}{$\lnot$ validPoint(p)$\implies$ IndexOutOfBoundsException}
\end{itemize}

\noindent get($p$):
\begin{itemize}
\item output: \textcolor{blue}{$out := s[p.row()][p.col()]$}
\item exception:\\ \textcolor{blue}{$\lnot$ validPoint(p)$\implies$ IndexOutOfBoundsException}
\end{itemize}

\noindent getNumRow():
\begin{itemize}
\item output: $out := \mbox{nRow}$
\item exception: None
\end{itemize}

\noindent getNumCol():
\begin{itemize}
\item output: $out := \mbox{nCol}$
\item exception: None
\end{itemize}

\subsection*{Local Functions}

\noindent validRow: $\mathbb{N} \rightarrow \mathbb{B}$\\
\noindent \textcolor{blue}{validRow($r$) $\equiv r \geq 0 \land (r < $ nRow)} \\

\noindent validCol: $\mathbb{N} \rightarrow \mathbb{B}$\\
\noindent \textcolor{blue}{validCol($c$) $\equiv (c \geq 0) \land (c < $ nCol)} \\


\noindent validPoint: $\mbox{PointT} \rightarrow \mathbb{B}$\\
\noindent \textcolor{blue}{validPoint($p$) $\equiv$ validCol(p.col()) $\land$ validRow(p.row())}

\newpage

\section* {BoardMoves Module}

\subsection* {Template Module}

\noindent \textcolor{blue}{BoardMoves is seq of PointT}

\subsubsection* {Considerations}
When implementing in Java. Extend Arraylist parameterized by the type PointT


\newpage

\section* {TwoDotsBoard Module}

\subsection* {Template Module}

\noindent \textcolor{blue}{TwoDotsBoard is Board(Color)}

\subsection* {Syntax}

\subsubsection* {Exported Constants}

None

\subsubsection* {Exported Access Programs}

\begin{tabular}{| l | l | l | p{6cm} |}
\hline
\textbf{Routine name} & \textbf{In} & \textbf{Out} & \textbf{Exceptions}\\
\hline
validateMoves & BoardMoves & $\mathbb{B}$& \\
\hline
updateBoard & BoardMoves &  & \\
\hline
\end{tabular}

\subsection* {Semantics}

\subsubsection* {Access Routine Semantics}

\noindent validateMoves(b): 
\begin{itemize}
\item \textcolor{blue}{output : out $:= |b| > 1$  $\land$  $\forall ( p : PointT$ $|$ $p$ $\in $b : validPoint(p)$) \land validPath(b) \land isDistinct(b)$}
\item exception: None
\end{itemize}

\noindent updateBoard(b): 
\begin{itemize}
\item \textcolor{blue}{output : out $ := $ None}
\item \textcolor{blue}{transition : s $ := \forall (p : PointT | p \in b  \forall(i : \mathbb{N} | i \in [p.row()..1]) : s[i][p.col()] := $ randomColor() ) }
\item exception: None
\end{itemize}

\subsection*{Local Functions}

\noindent validRow: $\mathbb{N} \rightarrow \mathbb{B}$\\
\noindent \textcolor{blue}{validRow($r$) $\equiv r \geq 0 \land (r < $ nRow)} \\

\noindent validCol: $\mathbb{N} \rightarrow \mathbb{B}$\\
\noindent \textcolor{blue}{validCol($c$) $\equiv (c \geq 0) \land (c < $ nCol)} \\


\noindent validPoint: $\mbox{PointT} \rightarrow \mathbb{B}$\\
\noindent \textcolor{blue}{validPoint($p$) $\equiv$ validCol(p.col()) $\land$ validRow(p.row())}\\

\noindent isDistinct: BoardMoves $\rightarrow \mathbb{B}$\\
\noindent isDistinct($b$) $\equiv$ $\forall(i : \mathbb{N}| i \in [0..|b|-1] : \forall (j : \mathbb{N} | j \in [(i+1)..|b|-1]) : \lnot (b[i].row() = b[j].row()) \land (b[i].col() = b[j].col()) )$  \\


\noindent validPath: $\mbox{BoardMoves} \rightarrow \mathbb{B}$\\
\noindent \textcolor{blue}{validPath($b$) $\equiv \forall(i : \mathbb{N} | i \in [0..|b|-2] : isAdjacent(b,i,i+1) \land sameColor(b,i,i+1)) $}\\
 
\noindent sameColor : BoardMoves $\times \mathbb{N} \times \mathbb{N} \rightarrow \mathbb{B} $\\
\noindent sameColor($b,i,j$) $\equiv s[b[i].row()][b[i].col()] = s[b[j].row()][b[j].col()]$\\

\noindent isAdjacent: BoardMoves $\times \mathbb{N} \times \mathbb{N} \rightarrow \mathbb{B}$\\
\noindent \textcolor{blue}{isAdjacent($b,i,j$) $\equiv b[i].row() = b[j].row() \land b[i].col() = b[j].col() + 1$\\
$\lor$  $b[i].row() = b[j].row() \land b[i].col() = b[j].col() - 1$\\
$\lor$  $b[i].row() = b[j].row() - 1 \land b[i].col() = b[j].col()$\\
$\lor$  $b[i].row() = b[j].row() + 1 \land b[i].col() = b[j].col()$ }\\

\subsection*{Considerations}
\noindent In Java, calling randomColor() represents a call to a static function so it would actually be done as Color.randomColor()


\newpage

\section* {Strategy Interface Module}

\subsection* {Interface Module}

\noindent {Strategy}

\subsection* {Syntax}

\subsubsection* {Exported Constants}

None

\subsubsection* {Exported types}

None

\subsubsection* {Exported Access Programs}

\begin{tabular}{| l | l | l | p{6cm} |}
\hline
\textbf{Routine name} & \textbf{In} & \textbf{Out} & \textbf{Exceptions}\\
\hline
play & TwoDotsBoard & & \\
\hline
\end{tabular}

\newpage

\section* {BoardView Module}

\subsection* {Template Module}

\noindent \textcolor{blue}{BoardView}

\subsection* {Syntax}

\subsubsection* {Exported Constants}

None

\subsubsection* {Exported Access Programs}

\begin{tabular}{| l | l | l | p{6cm} |}
\hline
\textbf{Routine name} & \textbf{In} & \textbf{Out} & \textbf{Exceptions}\\
\hline
printBoard & TwoDotsBoard & $\mathbb{B}$& \\
\hline
modePrompt &  & Strategy & \\
\hline
getInput &  & BoardMoves & \\
\hline
closeStream &  &  & \\
\hline
printMsg & $msg : string$ &  & \\
\hline

\end{tabular}

\subsection* {Semantics}

\subsubsection* {Environment variables}
\noindent $s$ : 2D sequence of pixels displayed on a standard Unix Shell/console\\
\noindent $r$ : an object to write text out on a standard Unix Shell/console\\

\subsubsection* {Access Routine Semantics}

\noindent printBoard($b$): 
\begin{itemize}
\item transition $s:=$ Modify the Console so that the TwoDotsBoard b is printed in a tabular manner. The contents of each row from $b$ should
be on individual line. Their should be horizontal and vertical numbering indicating each row and column from one upto and including the row and column size of the board
\item exception: None
\end{itemize}

\noindent modePrompt(): 
\begin{itemize}
\item transition :
\begin{itemize}
\item $s:=$ Modify the console to print a message asking the user to enter ``T'' for the timed version of the game and ``M'' for the mode of the game with a set number of moves
\item $r$:= read a single line of text from the standard input. Store this value in memory and then determine what to output as follows:\\
If the line read in is ``T'' or ``t'' output : out $ := $ new TimedStrategy()\\
If the line read in is ``M'' or ``m'' output: out $ := $ new MovesStrategy()\\
Otherwise keep reading a line from the standard input until one of the above two conditions are met
\end{itemize}
\item exception: None
\end{itemize}

\noindent getInput(): 
\begin{itemize}
\item transition :
\begin{itemize}
\item $r$:= read a single line of text from the standard input to determine the coordinates of the dots the user would like to eliminate.
Note that the desired input format is u,v w,x y,z .... These are pairs of natural numbers with a comma between them and each pair is separated by a space. Store this value in memory and then determine what to output as follows:\\
If the line read in is in the correct format then $output: out :=$ new BoardMoves() containing the pairs of integers\\
Otherwise keep reading a line from the standard input until one of the above conditions are met
\end{itemize}
\end{itemize}

\noindent closeStream(): 
\begin{itemize}
\item transition : $s$:= close the input stream
\end{itemize}

\noindent printMsg($msg : string$): 
\begin{itemize}
\item transition : $s$:= Modify the output console to print out text contain in the string $msg$
\end{itemize}

\subsection* {Considerations}
\noindent In java, closing the input stream corresponds to closing the System.in object


\newpage

\section* {BoardController Module}

\subsection* {Template Module}

\noindent {BoardController}

\subsection* {Uses}

TwoDotsBoard, BoardView, Color, PointT, BoardMoves

\subsection* {Syntax}

\subsubsection* {Exported Constants}

None

\subsubsection* {Exported Access Programs}

\begin{tabular}{| l | l | l | p{6cm} |}
\hline
\textbf{Routine name} & \textbf{In} & \textbf{Out} & \textbf{Exceptions}\\
\hline
BoardController & TwoDotsBoard, BoardView & BoardController& \\
\hline
get & PointT & Color & \\
\hline
set & PointT, Color &  & \\
\hline
validateMoves & BoardMoves & $\mathbb{B}$ & \\
\hline
updateBoard & BoardMoves &  & \\
\hline
updateView &   &  & \\
\hline
printMsg & $msg : string$ &  & \\
\hline
modePrompt &  & Strategy & \\
\hline
closeViewStream &  &  & \\
\hline
getInput &  & BoardMoves & \\
\hline

\end{tabular}

\subsection* {Semantics}


\subsubsection* {State variables}
\noindent $m$ : TwoDotsBoard
\noindent $v$ : BoardView



\subsubsection*{State invariant}
\noindent None

\subsubsection* {Access Routine Semantics}


\noindent BoardController($model$,$view$)
\begin{itemize} 
\item output: out := self
\item transition: $m:=$ model, $v:=$ view
\item exceptions: none
\end{itemize}

\noindent get(p)
\begin{itemize}
\item output : $out := $ m.get(p)
\item transition: none
\item exceptions: none
\end{itemize}

\noindent set(p,c)
\begin{itemize}
\item transition: m.set(p,c)
\item exception : none
\end{itemize}

\noindent validateMoves(b)
\begin{itemize}
\item output : $out := $ m.validateMoves(b)
\end{itemize}

\noindent updateBoard(b)
\begin{itemize}
\item transition: m.updateBoard(b)
\end{itemize}

\noindent updateView()
\begin{itemize}
\item transition: v.printBoard(m)
\end{itemize}

\noindent printMsg($msg : string$): 
\begin{itemize}
\item transition view.printMsg($msg$)
\end{itemize}


\noindent modePrompt(): 
\begin{itemize}
\item output: $out := $ v.modePrompt()
\end{itemize}

\noindent closeViewStream(): 
\begin{itemize}
\item transition :v.closeStream()
\end{itemize}


\noindent getInput(): 
\begin{itemize}
\item output: if (m.validateMoves(v.getInput())) then $out:=$ else $out :=$ getInput()
\end{itemize}


\newpage

\section* {StrategyGameMode Module}

\subsection* {Template Module inherits Strategy}

\noindent {StrategyGameMode}

\subsection*{Uses}
Strategy, BoardView, BoardController, BoardMoves, TwoDotsBoard

\subsection* {Syntax}

\subsubsection* {Exported Constants}

None

\subsubsection* {Exported Access Programs}

\begin{tabular}{| l | l | l | p{6cm} |}
\hline
\textbf{Routine name} & \textbf{In} & \textbf{Out} & \textbf{Exceptions}\\
\hline
play & TwoDotsBoard & & \\
\hline
startUp & TwoDotsBoard & & \\
\hline
checkWin & & $\mathbb{B}$ & \\
\hline
canContinue &  & $\mathbb{B}$ & \\
\hline
updateData &  & & \\
\hline
introMsg & & & \\
\hline
endMsg & & & \\
\hline
\end{tabular}

\subsection* {Semantics}


\subsubsection* {State variables}
\noindent $c$ : BoardController\\
\noindent $v$ : BoardView\\
\noindent $moves$ : BoardMoves\\


\subsubsection*{State invariant}
\noindent None

\subsubsection* {Access Routine Semantics}

\noindent play(b)
\begin{itemize}
\item transition: 
\begin{itemize}
\item startUp(b)
\item introMsg()
\item if canContinue() then:
\begin{itemize}
\item c.updateView()
\item c.updateBoard(c.getInput())
\item if checkWin() then 
\item updateData()
\item if canContinue() then repeat these steps labeled with $*$
\end{itemize} 
\end{itemize}
\end{itemize}

\subsection* {Consideration}
\noindent In Java, this module would be implemented as an abstract class that implements the Strategy interface. Unimplemented methods are ones that are abstract methods and will be overridden by its children\\
\noindent This is the best that could be done to convey the idea of a ``abstract class'' given that MIS does not have the notion of an abstract class, following Dr Smith's advice to use Inheritance and leave a note for reader. Source: \href{https://avenue.cllmcmaster.ca/d2l/le/296632/discussions/threads/1286234/View?groupFilterOption=0&searchText=abstract} {Here (will have to login to avenue)}

\newpage

\section* {MovesStrategy Module}

\subsection* {Template Module inherits StrategyGameMode}

\noindent {MovesStrategy}

\subsection*{Uses}
StrategyGameMode, BoardView, BoardController, BoardMoves, TwoDotsBoard

\subsection* {Syntax}

\subsubsection* {Exported Constants}

None

\subsubsection* {Exported Access Programs}

\begin{tabular}{| l | l | l | p{6cm} |}
\hline
\textbf{Routine name} & \textbf{In} & \textbf{Out} & \textbf{Exceptions}\\
\hline
play & TwoDotsBoard & & \\
\hline
startUp & TwoDotsBoard & & \\
\hline
checkWin & & $\mathbb{B}$ & \\
\hline
canContinue &  & $\mathbb{B}$ & \\
\hline
updateData &  & & \\
\hline
introMsg & & & \\
\hline
endMsg & & & \\
\hline
\end{tabular}

\subsection* {Semantics}

\subsubsection* {Environment variables}


\subsubsection* {State variables}
\noindent $moveCount$ : $\mathbb{N}$\\
\noindent $TARGET$ : $\mathbb{N}$\\


\subsubsection*{State invariant}
\noindent None

\subsubsection* {Access Routine Semantics}

\noindent startUp(b)
\begin{itemize}
\item transition: $v,c,moveCount,TARGET :=$  new BoardView(), new BoardController($b,v$), 15, 5 
\end{itemize}


\noindent checkWin()
\begin{itemize}
\item transition: if $|moves| < TARGET$ then $moves := moves - 1$
 \item outupt: $out :=$ if $|moves| >= TARGET$ then $out := true$ else $out := false$
\end{itemize}

\noindent canContinue()
\begin{itemize}
\item output: $out := moveCount > 0$
\end{itemize}


\noindent updateData()
\begin{itemize}
\item transition: if $moveCount = 0$ then c.exit() \\
else $c := $ using c.printMsg(), modify the screen to print out how many moves are left, i.e print the value of $moveCount$
\end{itemize}


\noindent introMsg()
\begin{itemize}
\item transition: $c :=$ using c.printMsg(), modify the screen to print a message to state the rules. This includes instruction for how to enter input, how to win and how many total moves the user has
\end{itemize}

\noindent endMsg()\\
\noindent transition: $c :=$ using c.printMsg(), modify the screen by printing a message telling the user the game is over


\newpage

\section* {CountDownTimer Module}

\subsection* {Module}

\noindent {CountDownTimer}

\subsection*{Uses}
None

\subsection* {Syntax}

\subsubsection* {Exported Constants}

None

\subsubsection* {Exported Access Programs}

\begin{tabular}{| l | l | l | p{6cm} |}
\hline
\textbf{Routine name} & \textbf{In} & \textbf{Out} & \textbf{Exceptions}\\
\hline
newTimer & $\mathbb{Z}$ & & IllegalArgumentException \\
\hline
isCancelled &  & $\mathbb{B}$ & \\
\hline
\end{tabular}

\subsection* {Semantics}

\subsubsection* {Environment variables}
\noindent $t$ : Represents the system clock\\



\subsubsection* {State variables}
\noindent $cancelled$ : $\mathbb{B}$\\
\noindent $multiplier$ : $\mathbb{Z}$\\


\subsubsection*{State invariant}
\noindent None

\subsubsection* {Access Routine Semantics}

\noindent newTimer(time : $\mathbb{Z}$)
\begin{itemize}
\item transition: $cancelled := false$\\
$multiplier := 1000$\\
$t := $ Use the system clock to start tracking the current time. Once time*multiplier amount of time has passed then the transition $cancelled := true$ happens\\
\item exception: $exc := $ time $< 1 \implies$ IllegalArgumentException
\end{itemize}

\noindent isCancelled()
\begin{itemize}
\item output: $out := \lnot cancelled$
\end{itemize}

\newpage

\section* {TimedStrategy Module}

\subsection* {Template Module inherits StrategyGameMode}

\noindent {TimedStrategy}

\subsection*{Uses}
Strategy, BoardController, StrategyGameMode, CountDownTimer

\subsection* {Syntax}

\subsubsection* {Exported Constants}

None

\subsubsection* {Exported Access Programs}

\begin{tabular}{| l | l | l | p{6cm} |}
\hline
\textbf{Routine name} & \textbf{In} & \textbf{Out} & \textbf{Exceptions}\\
\hline
play & TwoDotsBoard & & \\
\hline
startUp & TwoDotsBoard & & \\
\hline
checkWin & & $\mathbb{B}$ & \\
\hline
canContinue &  & $\mathbb{B}$ & \\
\hline
updateData &  & & \\
\hline
introMsg & & & \\
\hline
endMsg & & & \\
\hline
\end{tabular}

\subsection* {Semantics}

\subsubsection* {Environment variables}
\noindent $t$ : Represents the system clock\\



\subsubsection* {State variables}
\noindent $TARGET$ : $\mathbb{Z}$\\
\noindent $TIME$ : $\mathbb{Z}$\\


\subsubsection*{State invariant}
\noindent None

\subsubsection* {Access Routine Semantics}

\noindent startUp(b)
\begin{itemize}
\item transition: $v,c,moveCount,TARGET,TIME :=$  new BoardView(), new BoardController($b,v$), 5, 60\\
newTimer(TIME) 
\end{itemize}


\noindent checkWin()
\begin{itemize}
\item output: $out := |moves| >= TARGET$
\end{itemize}

\noindent canContinue()
\begin{itemize}
\item output: $out := $ isCancelled()
\end{itemize}


\noindent updateData()
\begin{itemize}
\item transition: $c := $ using c.printMsg(), modify the screen to print out how much time has elapsed since the start of the game. Formally, print out $t.now() - TIME$
\end{itemize}


\noindent introMsg()
\begin{itemize}
\item transition: $c :=$ using c.printMsg(), modify the screen to print a message to state the rules. This includes instruction for how to enter input, how to win and how much time the user has
\end{itemize}

\noindent endMsg()\\
\noindent transition: $c :=$ using c.printMsg(), modify the screen by printing a message telling the user the game is over

\subsection*{Consideration}
\noindent When calling methods such as newTimer() and isCancelled() these are references to the access programs defined in CountDownTimer module. In java these would static access program and would be accessed without creating an instance of the module. Also assume $t.now()$ gives the current time in seconds

\newpage

\section* {Dots Module}

\subsection* {Module}

\noindent {Dots}

\subsection*{Uses}
Strategy, BoardController, TwoDotsBoard, BoardView

\subsection* {Syntax}

\subsubsection* {Exported Constants}

None

\subsubsection* {Exported Access Programs}

\begin{tabular}{| l | l | l | p{6cm} |}
\hline
\textbf{Routine name} & \textbf{In} & \textbf{Out} & \textbf{Exceptions}\\
\hline
main & seq of $string$ & & \\
\hline
\end{tabular}

\subsection* {Semantics}

\subsubsection* {Environment variables}
None

\subsubsection* {State variables}
None

\subsubsection*{State invariant}
\noindent None

\subsubsection* {Access Routine Semantics}

\noindent main($args$ : seq of $string$):\\
\noindent transition:\\
new BoardController(new TwoDotsBoard(6,6), new BoardView()).modePrompt().play(new TwoDotsBoard(6,6))

\subsection*{Considerations}
The main role of this module, in terms of implementation is to instantiate the controller, start the desired mode the user wants and then play the game while manipulating the board and interacting with the user via the controller.

\newpage

\section*{Discussion Question}
\subsection*{Critique of design}
\begin{enumerate}
\item Consistency: 

\item Essentiality: In general I believe the design provided is essential for all the modules. However, essentially becomes more difficult to achieve the more complex a design becomes. To maintain essentiallity in all of the modules I did my best to modularize the design. Splitting different concerns of the game into different modules, ADT's and abstract objects. However, there are some parts of the design that are not essential and could be improved. 

\item Generality: The MIS/Interface design is general. For example, the PointT class can represent any point in 2D space not just a point on a TwoDotsBoard. The Board module is a general purpose 2D grid that can be customized with any dimensions and has some useful and general access programs that one may want to use on a 2D grid. The StrategyGameMode is also very general. It defines a very high level iterative process for playing a game of TwoDots, along with several abstract/template methods to check conditions as the game is played. When looked at it more closely the StrategyGameMode can be adopted to many different games, not just TwoDots. Any game that has a set number of moves, time or other conditions can be created by inheriting the StrategyGameMode module. 

\item Minimality: In general, I tried to keep all modules minimal and address only a single concern. Most of the modules are minimal and do not offer more than one service, however there are a few exception and this is the challenge with designing a large system while maintaining such qualities. One example of a function that returns a value and has a potential state transition depending on a boolean condition is the checkWin() method in MovesStrategy module. This module return whether the user just eliminated 5 or more dots in move. However, if the user did not achieve this target then the state variable ``moves'' counter is decremented and then the function return false. So, here we have a state transition and a output thus not minimal. In the future, to make this minimal this can be achieved by splitting the function into two functions. One that checks the winning condition only and another that decrements the ``moves'' variable only.

\item Cohesion: In general, all the modules have high cohesion. Every module only has methods and data structures/types that are related. For example the TwoDotsBoard module

\item Information hiding: I have kept the inner working details of my modules as encapsulated as possible. For example in my TwoDotsBoard module there are getter methods for the row and column. The randomColor() function is described as general as possible giving the flexbility of being implemented in any way. Same goes for the BoardView and CountDownTimer modules. These describe at a high level what should be outputted to the screen and how the timer should operate. In terms, of Java the timer may be implemented using the System.nanoTime() or other functions provided in System to access time or it may be done using the Timer class and set a Timer in the background as the game plays. The MIS, abstracts this all way and only requires a isCancelled() method to indicate whether a required amount of time has past or not.

\end{enumerate}
\end {document}