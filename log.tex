\documentclass[12pt]{article}

\usepackage{graphicx}
\usepackage{paralist}
\usepackage{listings}
\usepackage{booktabs}

\oddsidemargin 0mm
\evensidemargin 0mm
\textwidth 160mm
\textheight 200mm

\pagestyle {plain}
\pagenumbering{arabic}

\newcounter{stepnum}

\newcounter{weeknum}
\setcounter{weeknum}{1}

\newcounter{lectnum}
\setcounter{lectnum}{1}

\newcounter{tutnum}
\setcounter{tutnum}{1}

\title{Learning Log}
\author{Shazil Arif}
\date{\today}

\begin {document}

\maketitle

The purpose of the learning log is to reflect upon your progress in learning the
content of SE 2AA4/ CS 2ME3.  This is a personal journal.  The intention is for
you to be aware of your progress by means of recording and reflecting.  A
template is provided for each week.  You should fill in the question marks.  You
are also free to add your own subsections.

%%%%%%%%%%%%%%%%%%%%%%%%%%%%%%%%%%%%%%%%%%%%%%%%%%%%%%%%%%%%%%%%%%%%%%%%%%

\section {Week \theweeknum \addtocounter{weeknum}{1} Intro to Course}

\subsubsection *{Dates} Jan 6 to Jan 10

\subsubsection *{Lecture \thelectnum \addtocounter{lectnum}{1} Introduction to Course}

Discuss Course administrative details, marking scheme, material and content to be taught

\subsubsection *{Lecture \thelectnum \addtocounter{lectnum}{1} Software
  Engineering Profession}

Discussed the differences between studying Computer Science and Software Engineering, History of Software Engineering and some important figures such as Parnas

\subsubsection *{Tutorial \thetutnum \addtocounter{tutnum}{1} Git, Doxygen and A1}

Learnt how to install Doxygen, Tex, Git and set up necessary tools and development environment to complete assignments

\subsubsection *{Textbook Reading (Ghezzi, H\&S or other)}

I did not read the book this week

\subsubsection *{Assignment Progress}

Finished the coding and testing portions

\subsubsection *{Midterm/Final Review Progress}

Have not started

\subsubsection *{Reflection Relating Course Topics, Other Courses, Other Experiences}

The discussion board is quite helpful. I find that this course really helps you understand deeply what "Software Engineering" really is. It focuses on the practical aspects of the profession rather than just programming. 

%%%%%%%%%%%%%%%%%%%%%%%%%%%%%%%%%%%%%%%%%%%%%%%%%%%%%%%%%%%%%%%%%%%%%%%%%%

\section {Week \theweeknum \addtocounter{weeknum}{1} Software Qualities, Software Engineering Principles}

\subsubsection *{Dates} Jan 13 to Jan 17

\subsubsection *{Lecture \thelectnum \addtocounter{lectnum}{1} Software Qualities}

Discussed Software qualities such as correctness, robustness, reliability, portability, maintainibility, etc.. Comparing and contrasting different terminology

\subsubsection *{Lecture \thelectnum \addtocounter{lectnum}{1} Software Engineering principles}

Discussed key SE principles including abstraction, information hiding, designing for change, seperation of concerns and more

\subsubsection *{Tutorial \thetutnum \addtocounter{tutnum}{1} Basics of Latex and PEP8 convention for Python}

Learnt the basics of latex, syntax, different editors etc. Disucssed the PEP8 standards 

\subsubsection *{Textbook Reading (Ghezzi, H\&S or other)}

Have not started :-(

\subsubsection *{Assignment Progress}

Nearly complete. missing a few tests for pos adt

\subsubsection *{Midterm/Final Review Progress}

Have not started reviewing

\subsubsection *{Reflection Relating Course Topics, Other Courses, Other Experiences}

It is nice to see many different software engineering principles, methods and practices being defined in detail and how to apply them

%%%%%%%%%%%%%%%%%%%%%%%%%%%%%%%%%%%%%%%%%%%%%%%%%%%%%%%%%%%%%%%%%%%%%%%%%%

\section {Week \theweeknum \addtocounter{weeknum}{1} Introduction to modules and Mathematics for MIS }

\subsubsection *{Dates} Jan 20 to Jan 24

\subsubsection *{Lecture \thelectnum \addtocounter{lectnum}{1} Introduction to Modules}

Important goals to keep in mind when developing software such as Design for change and Product families.
Discussed The module interface, module implementation

Information Hiding:
Basis for design
Implementation secrets are hidden from clients
Encapsulate changeable design decisions as implementation secrets within module implementations
Encapsulate changeable design decisions as implementation secrets with module implementations

The WRONG ANS: HAS NOTHING TO DO WITH Security and HIDING DATA, VARIABLES 

Important for midterm! internalize it

Discussed examples of modules such as record, library, abstract data type, generic modules
Note: follow precise terminology from Ghezzi textbook

Difference between a library and module
\\
\\Library: Has no state information or record of any stored data. E.g a Math library that has functions that take inputs and gives outputs
\\
\\Module: Has state information and some record of data (a ADT module?)

When implementing a specification must match it, not look like it

\subsubsection *{Lecture \thelectnum \addtocounter{lectnum}{1} Mathematics for MIS}

Worked through an example of balancing chemical equations to demonstrate how we can take a problem, describe it in mathematical terms and syntax and from there translate to an actual program/code 

\subsubsection *{Lecture \thelectnum \addtocounter{lectnum}{1} Module Interface Specification}
Worked through an example of defining an abstract data type for a circle
\\Note : MIS is not giving the implementation, it only defines the interface!
e.g MIS may give a specification isbalanced() that returns whether an equation is balanced..But it is not specified how to achieve this, it us up to the developer to figure out the implementation!
\\
\\An abstract object is programming terms is a module where this only one instance/singleton pattern


\subsubsection *{Tutorial \thetutnum \addtocounter{tutnum}{1} Math Review}

Reviewed mathematical operators, unary and binary operators ad their precedences

Discussed what a set is:
1) Distinct elements (i.e no elements are repeated)
2) All elements are of the same type

Operations on sets:
Union: essentially combine two sets
Intersection: elements in both sets
Set Difference: Take first set and remove any elements that are common with other set 
e.g if we have S=  {1,2} and T = {2,3,4} then we have S - T = {1} . 3 and 4 not included cuz not in both sets

Subset

Cartesian product: all possible pairs

A set can be described in two ways:
set enumeration: List out all elements in a set

Set comprehension: 
$S = { x : t | R : E}$
This means S is a set where its elements are of type t and satisfy a property R and E is some defining expression for a set element
e.g $S = {x : N | 1 <= x < 5 : x ^2}$ then $S = {1,4,9,16}$

Types:
A set of values e.g a value of type integer belongs to the set  S = {...-1,-2,0,1,2....}

We can have custom types:
Such as a PointT type which can be a tuple(x : R, y : R)

Quantifiers (Shorthand for applying the same operator many times)

( *x : X | R : P)
x is an element of type X
R is a range (usually a boolean condition indicating which elements to include/consider)
P - the values to apply the operator "*" to.
* may be +,-, / etc.
e.g {+x : N | 1<= x < 5 : $x^2$} means to sum up the square of the terms from 1 to 5 (including 1 but not 5)

Quantifiers for conjunction and disjunction
for logical and, we use the universal quantifier forall, since "and"

for logical or, we use existential quantifier, exists since "or" 

\subsubsection *{Textbook Reading (Ghezzi, H\&S or other)}

Did not read 

\subsubsection *{Assignment Progress}

Complete

\subsubsection *{Midterm/Final Review Progress}

Reviewing principles

\subsubsection *{Reflection Relating Course Topics, Other Courses, Other Experiences}

Discussing modular design and information hiding tied in with what is currently being taught in our 2XB3 course, modular design and object oriented programming with Java. The overlapping material helps build a deeper understanding!

The Math review helps tie in with other courses such as Discrete Mathematics (2FA3). It is very interesting to see how to take a high level specification given in mathematical terminology and syntax and actually translating it to a program.

%%%%%%%%%%%%%%%%%%%%%%%%%%%%%%%%%%%%%%%%%%%%%%%%%%%%%%%%%%%%%%%%%%%%%%%%%%

\section {Week \theweeknum \addtocounter{weeknum}{1} Abstract Data Types}

\subsubsection *{Dates} Jan 27 to Jan 31

\subsubsection *{Lecture \thelectnum \addtocounter{lectnum}{1} Implementing ADT's in Python}

H and S notation $s[0:0]$ includes the upper bound. Python notation $s[0:n]$ goes up to n - 1\\
Looked at different examples of implementations of ADT's such as a (x,y) point, a line ADT\\
A mutator function will have a transition state (since they modify the state)\\
can use x,y,z point in 3d space to define latitude longitude positions, and use complex math to transform it into 2d coords\\
@staticmethod?\\

Implementation and specification are very different! Keep this in mind\\
Abstract object: Singleton, 
Abstract data type: Instantiable, can have multiple copies



\subsubsection *{Lecture \thelectnum \addtocounter{lectnum}{1} More on ADT's and Assignment Discussion}

Reviewed the difference between an abstract object and abstract data type.\\

Difference between a reference and actual value. Aliasing and pointers.\\

Generic Modules. What if we want a stack module for booleans, integers, stacks, strings?...\\

A state invariant is a proposition/expression that always evaluate to true. the state variable may change but the invariant holds true. Example for a generic stack module the invariant is size $<=$ MAXSIZE.\\

Module state machine

\subsubsection *{Lecture \thelectnum \addtocounter{lectnum}{1} A2 discussion and Classes, Interfaces in Python, UML}

Discussed Assignment 2 details and how to work with classes, interfaces and inherited classes in python.\\

UML. 

Difference between type and a class.
Types are known at compile time
Class of an object may only be known at run time



\subsubsection *{Tutorial \thetutnum \addtocounter{tutnum}{1} Mathematical notation, MIS}

Discussed the MIS.\\
Worked through last years Assignment 2 to aid in our assignment and translating MIS specification to code.\\

\subsubsection *{Textbook Reading (Ghezzi, H\&S or other)}

Did not read

\subsubsection *{Assignment Progress}

Finished Assignment 1 + report. Started working on Assignment 2

\subsubsection *{Midterm/Final Review Progress}

Did not review

\subsubsection *{Reflection Relating Course Topics, Other Courses, Other Experiences}

Not much to comment, understood what a invariant is which tied in with discrete mathematics concepts 
also understood a bit of referencing/aliasing in python and pointers which tied in with the C programming course.


%%%%%%%%%%%%%%%%%%%%%%%%%%%%%%%%%%%%%%%%%%%%%%%%%%%%%%%%%%%%%%%%%%%%%%%%%%

\section {Week \theweeknum \addtocounter{weeknum}{1} Functional Programming and More Abstraction, MIS stuff}

\subsubsection *{Dates} Feb 3 to Feb 7

\subsubsection *{Lecture \thelectnum \addtocounter{lectnum}{1} Functional programming in Python}

More on UML, MIS, interfaces and abstraction.\\
Discussed more about generic types, answered some MC questions
Discussed some functional programming, same some live code examples of functional programming in python

\subsubsection *{Lecture \thelectnum \addtocounter{lectnum}{1} ?}

Did not attend

\subsubsection *{Lecture \thelectnum \addtocounter{lectnum}{1} Modules with External Interaction}

reduce and lambda (anonymous functions) in Python\\


\subsubsection *{Tutorial \thetutnum \addtocounter{tutnum}{1} Unit testing in python with pytest}

Blackbox vs whitebox testing.\\

Unit testing allows to identifying bugs in code

pytest\\

all tests are in a class\\
each test has its own method

setup method(self,method):\\
do something before every test. e.g. define some state variables

teardown method(self,method)\\
 do something after every test e.g.reset state variables

assert
pytest.approx
pytest.raise(exception name)\\


Have acceptable code coverage


\subsubsection *{Textbook Reading (Ghezzi, H\&S or other)}

Did not read

\subsubsection *{Assignment Progress}

Almost complete Assignment 2

\subsubsection *{Midterm/Final Review Progress}

None

\subsubsection *{Reflection Relating Course Topics, Other Courses, Other Experiences}

Interesting to see the concepts of interfaces and generic modules tie in with other courses such as 2XB3

%%%%%%%%%%%%%%%%%%%%%%%%%%%%%%%%%%%%%%%%%%%%%%%%%%%%%%%%%%%%%%%%%%%%%%%%%%

\section {Week \theweeknum \addtocounter{weeknum}{1} Assumptions, Exceptions, Module Interface Design}

\subsubsection *{Dates} Feb 10 to Feb 14

\subsubsection *{Lecture \thelectnum \addtocounter{lectnum}{1} Module Interface Design }

Assumptions vs Exceptions\\
For assignment 2 report refer to slide 7/20 in Lec 16 for "critique the design"\\

Quality Criteria of a modules \emph{interface}

Essential - Omit unnecessary features, functions etc, no redundant features\\
If asked if a module interface is essential, need to answer "can i remove features and still meet the specifications", whether or not a feature is a good idea/design choice is not the question.\\

Slide 10/20 Lec 16 , Tuple Idiom Version - too many setters

Minimal - more than one state transition -> not minimal.
		state transition and printing something. Avoid independent services .\\
		
General - e.g. functional programming makes interfaces more general.\\
Opaque - Satisfies information hiding.\\
 


\subsubsection *{Lecture \thelectnum \addtocounter{lectnum}{1} ?}

Did not attend due to a busy week. midterms + assignments

\subsubsection *{Tutorial \thetutnum \addtocounter{tutnum}{1} ?}

Did not attend, same reason as lecture above

\subsubsection *{Textbook Reading (Ghezzi, H\&S or other)}

Read about object oriented design

\subsubsection *{Assignment Progress}

Working on the report for assignment 2!

\subsubsection *{Midterm/Final Review Progress}

Have not started

\subsubsection *{Reflection Relating Course Topics, Other Courses, Other Experiences}

Not much to reflect this week since I did not attend 2/3 lectures and the tutorial. Not something that I will continue tho!

%%%%%%%%%%%%%%%%%%%%%%%%%%%%%%%%%%%%%%%%%%%%%%%%%%%%%%%%%%%%%%%%%%%%%%%%%%

\section {Midterm Break}

\subsubsection *{Dates} Feb 17 to Feb 21


%%%%%%%%%%%%%%%%%%%%%%%%%%%%%%%%%%%%%%%%%%%%%%%%%%%%%%%%%%%%%%%%%%%%%%%%%%

\section {Week \theweeknum \addtocounter{weeknum}{1} }

\subsubsection *{Dates} Feb 24 to Feb 28

\subsubsection *{Lecture \thelectnum \addtocounter{lectnum}{1} Maze Tracing Robot Example}

Steps in design:

\begin{enumerate}

\item Anticipated changes

\item No template keyword = not an ADT but an astract object

\item Some access routines are redundant because they will be only called once (e.g reading file and populating the maze)

\item slide 16/23 (Maze Tracing Robot Example) the exception statement is an ``if-else statement''

\item Potential Midterm question slide 19/23 (Maze Tracing Robot Example)

\end{enumerate}


\subsubsection *{Lecture \thelectnum \addtocounter{lectnum}{1} Specification via UML}
\begin{enumerate}
\item Generics and interfaces in Java
\item extends keyword
\item objects are compared via their reference (location in memory)

\end{enumerate}


\subsubsection *{Lecture \thelectnum \addtocounter{lectnum}{1} Specification via UML}
\begin{enumerate}
\item More on UML diagram
\item Measurable interface can be implemented and used by BankAccountT and PointInterfaceT
\item strategy is something you can change at run time 
\end{enumerate}



\subsubsection *{Tutorial \thetutnum \addtocounter{tutnum}{1} A1 solution and Midterm Review}

\begin{enumerate}
\item Discussed A1 and common mistakes
\item worked through 2017 midterm with a polling setup to vote for answers.//

2017 Midterm answers and explanations

	\begin{enumerate}
	\item false because x was mutated and then the UPDATED value was used when calculating y
	\item False
	\item B
	\item B, key idea is that we want to keep the same external interface so only getters need to change
	\item D, option A and C do the same thing
	\item B, Always false
	\item Q7 ?
	\item Q8, Answer: E, reduce takes two arguments: 1) a function with two parameters(binary function) 2:) a list that the binary function is applied to
	\item B, False
	\item Q10, True, If any access programs behavior can be achieved by a combination of any other program, invariant etc. then we can remove it (probably)
	\item C, D is  wrong because every implementation should be independent of the interface
	\item C, it is possible to keep the same number of access routines, e.g. if we split up one but dont add a routine for all the independent services
	\item D, fast for set membership because we can do binary search. C is false because it is an implementation detail does not concern with the fact order in a set matters
	\item Answer: E, Read up on Parnas Style Diagram?
	\item Lookup Book notation for list slicing, Answer: C
	\item Book slice notation includes the last element A[1:n] includes n. Set comprehension ${x: T | Condition: Expression} so in this case we are adding j based on given condtion, so $
	\item A, Q19, env variable is used when something external to the software is changed, like the output monitor or speaker etc.
	\item True, Because the ``uses'' module is used within maze
	\item Q22, D because B and C are the same thing
	\item D is true, why is C false?
	\item Q24, Answer: E
	\item Q25, Equality defined if two objects point to same location in memory, Answer: A
	\item Q26, Read carefully ``return set of cells'' thus C and D are wrong
	
	\end{enumerate}
\end{enumerate}



\subsubsection *{Textbook Reading (Ghezzi, H\&S or other)}

Reviewing parnas papers, lectures slides and Ghezzi textbook chapters for midterm

\subsubsection *{Assignment Progress}

Working on the A3 specification


\subsubsection *{Midterm/Final Review Progress}

Worked on 2017 midterm, reviewing lecture and topics covered for Thursday March 5

\subsubsection *{Reflection Relating Course Topics, Other Courses, Other Experiences}

Not much to reflect, interfaces and classes were covered in the 2xb3 course, which was the focus of this course for this week, so it was a nice reinforcement of what was already learned. Also interesting to compare and contrast python and Java and differences when implementing generic interfaces due to dynamic vs static typing

%%%%%%%%%%%%%%%%%%%%%%%%%%%%%%%%%%%%%%%%%%%%%%%%%%%%%%%%%%%%%%%%%%%%%%%%%%

\section {Week \theweeknum \addtocounter{weeknum}{1} Midterm Exam Week}

\subsubsection *{Dates} Mar 2 to Mar 6

\subsubsection *{Lecture \thelectnum \addtocounter{lectnum}{1} Midterm Review}

\begin{enumerate}
\item Passive vs active learning
\item Local functions do not have to be implemented in code
\item Ghezzi Chapter 1 to 4, Hoffman and Strooper (Ch 3,6,7, Ch1,2)
\item lectures and tutorials
\item Emphasis on lecture concepts
\item Faked rational design process (important)
\item Dynamic binding Some types are unknown and can change at runtime
\end{enumerate}



\subsubsection *{Lecture \thelectnum \addtocounter{lectnum}{1} Design pattern}

\begin{enumerate}
\item In UML italicized method names are abstract
\item Arrow indicates inheritance relation
\item Can add fly() so all ducks can inherit it
\item Can be a bad solution because fly() needs to be modified for many subclasses
\item Can potentially add a interface
\item Dotted line is implements (for a interface), solid is extends/inherits (superclass)
\item Inheritance is usually a "is a" relationship. E.g. Car is a Vehicle
\item Composition instead is "has a" relationship.  E.g. Duck has a Fly method
\end{enumerate}

\subsubsection *{Lecture \thelectnum \addtocounter{lectnum}{1} Design Patterns continued}

\begin{enumerate}
\item WE are not reusing a design, we are reusing a pattern
\item Abstract object, Generic abstract objects, ADT, Generic ADT's, Interface, Libraries, Records
\item Slide 7/37 Lecture ? Exam question. Ans: True
\item Strategy Design pattern
\item MVC. Seperate computational elements from I/O elements.

%MVC
\begin{enumerate}
\item The UI/ User Interface. Displaying data etc. User interacts with view
\item Handles changes/actions , the application /business logic
\item Model changes/updates data
\end{enumerate}

\item 3 types of Design patterns

\begin{enumerate}
\item Creational Design pattern
\item Structural Design pattern
\item Behavioral Design pattern
\end{enumerate}

\item Proxy Pattern. There is a middle man that gets the required data/result for a client but pretend as if it is being accessed directly. Example getting data from a embedded device from web interface through some IoT/http protocol.
\item  a REST API is proxy pattern?

\item Client $=>$ Proxy $=>$ Real Subject

\item Adapter Pattern

\begin{enumerate}
\item Motivated by the problem of different wall outlets, changing adapter. Only the physical interface matters. Power is not an issue
\item 
\end{enumerate}

\item ``Singleton'' Pattern. For abstract objects

\end{enumerate}


\subsubsection *{Tutorial \thetutnum \addtocounter{tutnum}{1} Java concepts}

\begin{enumerate}
\item Doxygen comments in java \begin{verbatim}
/**
* Author: Name
* Revised : Data
* 
* Description: text
*/

/**
* @brief
* @param
* @return 
* ...
*/
\end{verbatim}

\item Exceptions are thrown with throw new exceptionname(string Description)
\item Worked through last years excercises

\item module called Test* in file named Test*.java
\item import org.junit.*
\item import static org.junit.Assert.*;
\item can create global variables to be used in test
\item setUp() is JUnit equivalent to setup method, set global variables here
\begin{verbatim}
@before
setUp(){
}
\end{verbatim}
\item Every method begins with @Test
\item assertTrue(boolean value)
\item assertFalse(boolean value)
\item assertEquals(t1,t2);
\item JUnit documentation (T08 slide 17/22)
\item assertEquals(t1,t2, TOLERANCE)
\item 

\end{enumerate}

\subsubsection *{Textbook Reading (Ghezzi, H\&S or other)}

Did not read

\subsubsection *{Assignment Progress}

Completed Part 1 the Spec and Critique Questions

\subsubsection *{Midterm/Final Review Progress}

Reviewing for midterm

\subsubsection *{Reflection Relating Course Topics, Other Courses, Other Experiences}

?

%%%%%%%%%%%%%%%%%%%%%%%%%%%%%%%%%%%%%%%%%%%%%%%%%%%%%%%%%%%%%%%%%%%%%%%%%%

\section {Week \theweeknum \addtocounter{weeknum}{1} ?}

\subsubsection *{Dates} Mar 9 to Mar 13

\subsubsection *{Lecture \thelectnum \addtocounter{lectnum}{1} Intro to specification}

\begin{enumerate}
\item Adapter pattern?
\item Specification is a broad term meaning definition
\item Specification document should be maintainable, reusable and usable
\item Specifications should be:
\begin{enumerate}
\item Clear, unambigious and understandable
\item Consistent, complete (internal and external)
\item Incremental
\item Validatable
\item Abstract
\end{enumerate}
\end{enumerate}


\subsubsection *{Lecture \thelectnum \addtocounter{lectnum}{1} ?}

?

\subsubsection *{Tutorial \thetutnum \addtocounter{tutnum}{1} ?}

?

\subsubsection *{Textbook Reading (Ghezzi, H\&S or other)}

?

\subsubsection *{Assignment Progress}

?

\subsubsection *{Midterm/Final Review Progress}

?

\subsubsection *{Reflection Relating Course Topics, Other Courses, Other Experiences}

?

%%%%%%%%%%%%%%%%%%%%%%%%%%%%%%%%%%%%%%%%%%%%%%%%%%%%%%%%%%%%%%%%%%%%%%%%%%

\section {Week \theweeknum \addtocounter{weeknum}{1} ?}

\subsubsection *{Dates} Mar 16 to Mar 20

\subsubsection *{Lecture \thelectnum \addtocounter{lectnum}{1} ?}

?

\subsubsection *{Lecture \thelectnum \addtocounter{lectnum}{1} ?}

?

\subsubsection *{Tutorial \thetutnum \addtocounter{tutnum}{1} ?}

?

\subsubsection *{Textbook Reading (Ghezzi, H\&S or other)}

?

\subsubsection *{Assignment Progress}

?

\subsubsection *{Midterm/Final Review Progress}

?

\subsubsection *{Reflection Relating Course Topics, Other Courses, Other Experiences}

?

%%%%%%%%%%%%%%%%%%%%%%%%%%%%%%%%%%%%%%%%%%%%%%%%%%%%%%%%%%%%%%%%%%%%%%%%%%

\section {Week \theweeknum \addtocounter{weeknum}{1} ?}

\subsubsection *{Dates} Mar 23 to Mar 27

\subsubsection *{Lecture \thelectnum \addtocounter{lectnum}{1} ?}

?

\subsubsection *{Lecture \thelectnum \addtocounter{lectnum}{1} ?}

?

\subsubsection *{Tutorial \thetutnum \addtocounter{tutnum}{1} ?}

?

\subsubsection *{Textbook Reading (Ghezzi, H\&S or other)}

?

\subsubsection *{Assignment Progress}

?

\subsubsection *{Midterm/Final Review Progress}

?

\subsubsection *{Reflection Relating Course Topics, Other Courses, Other Experiences}

?

%%%%%%%%%%%%%%%%%%%%%%%%%%%%%%%%%%%%%%%%%%%%%%%%%%%%%%%%%%%%%%%%%%%%%%%%%%

\section {Week \theweeknum \addtocounter{weeknum}{1} ?}

\subsubsection *{Dates} Mar 30 to Apr 3

\subsubsection *{Lecture \thelectnum \addtocounter{lectnum}{1} ?}

?

\subsubsection *{Lecture \thelectnum \addtocounter{lectnum}{1} ?}

?

\subsubsection *{Tutorial \thetutnum \addtocounter{tutnum}{1} ?}

NA

\subsubsection *{Textbook Reading (Ghezzi, H\&S or other)}

?

\subsubsection *{Assignment Progress}

?

\subsubsection *{Midterm/Final Review Progress}

?

\subsubsection *{Reflection Relating Course Topics, Other Courses, Other Experiences}

?

%%%%%%%%%%%%%%%%%%%%%%%%%%%%%%%%%%%%%%%%%%%%%%%%%%%%%%%%%%%%%%%%%%%%%%%%%%

\section {Week \theweeknum \addtocounter{weeknum}{1} ?}

\subsubsection *{Dates} Apr 6 to Apr 7

\subsubsection *{Lecture \thelectnum \addtocounter{lectnum}{1} ?}

?

\subsubsection *{Tutorial \thetutnum \addtocounter{tutnum}{1} ?}

NA

\subsubsection *{Textbook Reading (Ghezzi, H\&S or other)}

?

\subsubsection *{Assignment Progress}

?

\subsubsection *{Midterm/Final Review Progress}

?

\subsubsection *{Reflection Relating Course Topics, Other Courses, Other Experiences}

?

%%%%%%%%%%%%%%%%%%%%%%%%%%%%%%%%%%%%%%%%%%%%%%%%%%%%%%%%%%%%%%%%%%%%%%%%%%

\end{document}