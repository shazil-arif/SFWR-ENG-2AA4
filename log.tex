\documentclass[12pt]{article}

\usepackage{graphicx}
\usepackage{paralist}
\usepackage{listings}
\usepackage{booktabs}

\oddsidemargin 0mm
\evensidemargin 0mm
\textwidth 160mm
\textheight 200mm

\pagestyle {plain}
\pagenumbering{arabic}

\newcounter{stepnum}

\newcounter{weeknum}
\setcounter{weeknum}{1}

\newcounter{lectnum}
\setcounter{lectnum}{1}

\newcounter{tutnum}
\setcounter{tutnum}{1}

\title{Learning Log}
\author{Shazil Arif}
\date{\today}

\begin {document}

\maketitle

The purpose of the learning log is to reflect upon your progress in learning the
content of SE 2AA4/ CS 2ME3.  This is a personal journal.  The intention is for
you to be aware of your progress by means of recording and reflecting.  A
template is provided for each week.  You should fill in the question marks.  You
are also free to add your own subsections.

%%%%%%%%%%%%%%%%%%%%%%%%%%%%%%%%%%%%%%%%%%%%%%%%%%%%%%%%%%%%%%%%%%%%%%%%%%

\section {Week \theweeknum \addtocounter{weeknum}{1} Intro to Course}

\subsubsection *{Dates} Jan 6 to Jan 10

\subsubsection *{Lecture \thelectnum \addtocounter{lectnum}{1} Introduction to Course}

Discuss Course administrative details, marking scheme, material and content to be taught

\subsubsection *{Lecture \thelectnum \addtocounter{lectnum}{1} Software
  Engineering Profession}

Discussed the differences between studying Computer Science and Software Engineering, History of Software Engineering and some important figures such as Parnas

\subsubsection *{Tutorial \thetutnum \addtocounter{tutnum}{1} Git, Doxygen and A1}

Learnt how to install Doxygen, Tex, Git and set up necessary tools and development environment to complete assignments

\subsubsection *{Textbook Reading (Ghezzi, H\&S or other)}

I did not read the book this week

\subsubsection *{Assignment Progress}

Finished the coding and testing portions

\subsubsection *{Midterm/Final Review Progress}

Have not started

\subsubsection *{Reflection Relating Course Topics, Other Courses, Other Experiences}

The discussion board is quite helpful. I find that this course really helps you understand deeply what "Software Engineering" really is. It focuses on the practical aspects of the profession rather than just programming. 

%%%%%%%%%%%%%%%%%%%%%%%%%%%%%%%%%%%%%%%%%%%%%%%%%%%%%%%%%%%%%%%%%%%%%%%%%%

\section {Week \theweeknum \addtocounter{weeknum}{1} Software Qualities, Software Engineering Principles}

\subsubsection *{Dates} Jan 13 to Jan 17

\subsubsection *{Lecture \thelectnum \addtocounter{lectnum}{1} Software Qualities}

Discussed Software qualities such as correctness, robustness, reliability, portability, maintainibility, etc.. Comparing and contrasting different terminology

\subsubsection *{Lecture \thelectnum \addtocounter{lectnum}{1} Software Engineering principles}

Discussed key SE principles including abstraction, information hiding, designing for change, seperation of concerns and more

\subsubsection *{Tutorial \thetutnum \addtocounter{tutnum}{1} Basics of Latex and PEP8 convention for Python}

Learnt the basics of latex, syntax, different editors etc. Disucssed the PEP8 standards 

\subsubsection *{Textbook Reading (Ghezzi, H\&S or other)}

Have not started :-(

\subsubsection *{Assignment Progress}

Nearly complete. missing a few tests for pos adt

\subsubsection *{Midterm/Final Review Progress}

Have not started reviewing

\subsubsection *{Reflection Relating Course Topics, Other Courses, Other Experiences}

It is nice to see many different software engineering principles, methods and practices being defined in detail and how to apply them

%%%%%%%%%%%%%%%%%%%%%%%%%%%%%%%%%%%%%%%%%%%%%%%%%%%%%%%%%%%%%%%%%%%%%%%%%%

\section {Week \theweeknum \addtocounter{weeknum}{1} Introduction to modules and Mathematics for MIS }

\subsubsection *{Dates} Jan 20 to Jan 24

\subsubsection *{Lecture \thelectnum \addtocounter{lectnum}{1} Introduction to Modules}

Important goals to keep in mind when developing software such as Design for change and Product families.
Discussed The module interface, module implementation

Information Hiding:
Basis for design
Implementation secrets are hidden from clients
Encapsulate changeable design decisions as implementation secrets within module implementations
Encapsulate changeable design decisions as implementation secrets with module implementations

The WRONG ANS: HAS NOTHING TO DO WITH Security and HIDING DATA, VARIABLES 

Important for midterm! internalize it

Discussed examples of modules such as record, library, abstract data type, generic modules
Note: follow precise terminology from Ghezzi textbook

Difference between a library and module
\\
\\Library: Has no state information or record of any stored data. E.g a Math library that has functions that take inputs and gives outputs
\\
\\Module: Has state information and some record of data (a ADT module?)

When implementing a specification must match it, not look like it

\subsubsection *{Lecture \thelectnum \addtocounter{lectnum}{1} Mathematics for MIS}

Worked through an example of balancing chemical equations to demonstrate how we can take a problem, describe it in mathematical terms and syntax and from there translate to an actual program/code 

\subsubsection *{Lecture \thelectnum \addtocounter{lectnum}{1} Module Interface Specification}
Worked through an example of defining an abstract data type for a circle
\\Note : MIS is not giving the implementation, it only defines the interface!
e.g MIS may give a specification isbalanced() that returns whether an equation is balanced..But it is not specified how to achieve this, it us up to the developer to figure out the implementation!
\\
\\An abstract object is programming terms is a module where this only one instance/singleton pattern


\subsubsection *{Tutorial \thetutnum \addtocounter{tutnum}{1} Math Review}

Reviewed mathematical operators, unary and binary operators ad their precedences

Discussed what a set is:
1) Distinct elements (i.e no elements are repeated)
2) All elements are of the same type

Operations on sets:
Union: essentially combine two sets
Intersection: elements in both sets
Set Difference: Take first set and remove any elements that are common with other set 
e.g if we have S=  {1,2} and T = {2,3,4} then we have S - T = {1} . 3 and 4 not included cuz not in both sets

Subset

Cartesian product: all possible pairs

A set can be described in two ways:
set enumeration: List out all elements in a set

Set comprehension: 
S = { x : t | R : E}
This means S is a set where its elements are of type t and satisfy a property R and E is some defining expression for a set element
e.g S = {x : N | 1 <= x < 5 : $x ^2$} then S = {1,4,9,16}

Types:
A set of values e.g a value of type integer belongs to the set  S = {...-1,-2,0,1,2....}

We can have custom types:
Such as a PointT type which can be a tuple(x : R, y : R)

Quantifiers (Shorthand for applying the same operator many times)

( *x : X | R : P)
x is an element of type X
R is a range (usually a boolean condition indicating which elements to include/consider)
P - the values to apply the operator "*" to.
* may be +,-, / etc.
e.g {+x : N | 1<= x < 5 : $x^2$} means to sum up the square of the terms from 1 to 5 (including 1 but not 5)

Quantifiers for conjunction and disjunction
for logical and, we use the universal quantifier forall, since "and"

for logical or, we use existential quantifier, exists since "or" 

\subsubsection *{Textbook Reading (Ghezzi, H\&S or other)}

?

\subsubsection *{Assignment Progress}

Complete

\subsubsection *{Midterm/Final Review Progress}

Reviewing principles

\subsubsection *{Reflection Relating Course Topics, Other Courses, Other Experiences}

Discussing modular design and information hiding tied in with what is currently being taught in our 2XB3 course, modular design and object oriented programming with Java. The overlapping material helps build a deeper understanding!

The Math review helps tie in with other courses such as Discrete Mathematics (2FA3)

%%%%%%%%%%%%%%%%%%%%%%%%%%%%%%%%%%%%%%%%%%%%%%%%%%%%%%%%%%%%%%%%%%%%%%%%%%

\section {Week \theweeknum \addtocounter{weeknum}{1} ?}

\subsubsection *{Dates} Jan 27 to Jan 31

\subsubsection *{Lecture \thelectnum \addtocounter{lectnum}{1} ?}

?

\subsubsection *{Lecture \thelectnum \addtocounter{lectnum}{1} ?}

?

\subsubsection *{Tutorial \thetutnum \addtocounter{tutnum}{1} ?}

?

\subsubsection *{Textbook Reading (Ghezzi, H\&S or other)}

?

\subsubsection *{Assignment Progress}

?

\subsubsection *{Midterm/Final Review Progress}

?

\subsubsection *{Reflection Relating Course Topics, Other Courses, Other Experiences}

?

%%%%%%%%%%%%%%%%%%%%%%%%%%%%%%%%%%%%%%%%%%%%%%%%%%%%%%%%%%%%%%%%%%%%%%%%%%

\section {Week \theweeknum \addtocounter{weeknum}{1} ?}

\subsubsection *{Dates} Feb 3 to Feb 7

\subsubsection *{Lecture \thelectnum \addtocounter{lectnum}{1} ?}

?

\subsubsection *{Lecture \thelectnum \addtocounter{lectnum}{1} ?}

?

\subsubsection *{Tutorial \thetutnum \addtocounter{tutnum}{1} ?}

?

\subsubsection *{Textbook Reading (Ghezzi, H\&S or other)}

?

\subsubsection *{Assignment Progress}

?

\subsubsection *{Midterm/Final Review Progress}

?

\subsubsection *{Reflection Relating Course Topics, Other Courses, Other Experiences}

?

%%%%%%%%%%%%%%%%%%%%%%%%%%%%%%%%%%%%%%%%%%%%%%%%%%%%%%%%%%%%%%%%%%%%%%%%%%

\section {Week \theweeknum \addtocounter{weeknum}{1} ?}

\subsubsection *{Dates} Feb 10 to Feb 14

\subsubsection *{Lecture \thelectnum \addtocounter{lectnum}{1} ?}

?

\subsubsection *{Lecture \thelectnum \addtocounter{lectnum}{1} ?}

?

\subsubsection *{Tutorial \thetutnum \addtocounter{tutnum}{1} ?}

?

\subsubsection *{Textbook Reading (Ghezzi, H\&S or other)}

?

\subsubsection *{Assignment Progress}

?

\subsubsection *{Midterm/Final Review Progress}

?

\subsubsection *{Reflection Relating Course Topics, Other Courses, Other Experiences}

?

%%%%%%%%%%%%%%%%%%%%%%%%%%%%%%%%%%%%%%%%%%%%%%%%%%%%%%%%%%%%%%%%%%%%%%%%%%

\section {Midterm Break}

\subsubsection *{Dates} Feb 17 to Feb 21


%%%%%%%%%%%%%%%%%%%%%%%%%%%%%%%%%%%%%%%%%%%%%%%%%%%%%%%%%%%%%%%%%%%%%%%%%%

\section {Week \theweeknum \addtocounter{weeknum}{1} ?}

\subsubsection *{Dates} Feb 24 to Feb 28

\subsubsection *{Lecture \thelectnum \addtocounter{lectnum}{1} ?}

?

\subsubsection *{Lecture \thelectnum \addtocounter{lectnum}{1} ?}

?

\subsubsection *{Tutorial \thetutnum \addtocounter{tutnum}{1} ?}

?

\subsubsection *{Textbook Reading (Ghezzi, H\&S or other)}

?

\subsubsection *{Assignment Progress}

?

\subsubsection *{Midterm/Final Review Progress}

?

\subsubsection *{Reflection Relating Course Topics, Other Courses, Other Experiences}

?

%%%%%%%%%%%%%%%%%%%%%%%%%%%%%%%%%%%%%%%%%%%%%%%%%%%%%%%%%%%%%%%%%%%%%%%%%%

\section {Week \theweeknum \addtocounter{weeknum}{1} Midterm Exam Week}

\subsubsection *{Dates} Mar 2 to Mar 6

\subsubsection *{Lecture \thelectnum \addtocounter{lectnum}{1} ?}

?

\subsubsection *{Lecture \thelectnum \addtocounter{lectnum}{1} ?}

?

\subsubsection *{Tutorial \thetutnum \addtocounter{tutnum}{1} ?}

?

\subsubsection *{Textbook Reading (Ghezzi, H\&S or other)}

?

\subsubsection *{Assignment Progress}

?

\subsubsection *{Midterm/Final Review Progress}

?

\subsubsection *{Reflection Relating Course Topics, Other Courses, Other Experiences}

?

%%%%%%%%%%%%%%%%%%%%%%%%%%%%%%%%%%%%%%%%%%%%%%%%%%%%%%%%%%%%%%%%%%%%%%%%%%

\section {Week \theweeknum \addtocounter{weeknum}{1} ?}

\subsubsection *{Dates} Mar 9 to Mar 13

\subsubsection *{Lecture \thelectnum \addtocounter{lectnum}{1} ?}

?

\subsubsection *{Lecture \thelectnum \addtocounter{lectnum}{1} ?}

?

\subsubsection *{Tutorial \thetutnum \addtocounter{tutnum}{1} ?}

?

\subsubsection *{Textbook Reading (Ghezzi, H\&S or other)}

?

\subsubsection *{Assignment Progress}

?

\subsubsection *{Midterm/Final Review Progress}

?

\subsubsection *{Reflection Relating Course Topics, Other Courses, Other Experiences}

?

%%%%%%%%%%%%%%%%%%%%%%%%%%%%%%%%%%%%%%%%%%%%%%%%%%%%%%%%%%%%%%%%%%%%%%%%%%

\section {Week \theweeknum \addtocounter{weeknum}{1} ?}

\subsubsection *{Dates} Mar 16 to Mar 20

\subsubsection *{Lecture \thelectnum \addtocounter{lectnum}{1} ?}

?

\subsubsection *{Lecture \thelectnum \addtocounter{lectnum}{1} ?}

?

\subsubsection *{Tutorial \thetutnum \addtocounter{tutnum}{1} ?}

?

\subsubsection *{Textbook Reading (Ghezzi, H\&S or other)}

?

\subsubsection *{Assignment Progress}

?

\subsubsection *{Midterm/Final Review Progress}

?

\subsubsection *{Reflection Relating Course Topics, Other Courses, Other Experiences}

?

%%%%%%%%%%%%%%%%%%%%%%%%%%%%%%%%%%%%%%%%%%%%%%%%%%%%%%%%%%%%%%%%%%%%%%%%%%

\section {Week \theweeknum \addtocounter{weeknum}{1} ?}

\subsubsection *{Dates} Mar 23 to Mar 27

\subsubsection *{Lecture \thelectnum \addtocounter{lectnum}{1} ?}

?

\subsubsection *{Lecture \thelectnum \addtocounter{lectnum}{1} ?}

?

\subsubsection *{Tutorial \thetutnum \addtocounter{tutnum}{1} ?}

?

\subsubsection *{Textbook Reading (Ghezzi, H\&S or other)}

?

\subsubsection *{Assignment Progress}

?

\subsubsection *{Midterm/Final Review Progress}

?

\subsubsection *{Reflection Relating Course Topics, Other Courses, Other Experiences}

?

%%%%%%%%%%%%%%%%%%%%%%%%%%%%%%%%%%%%%%%%%%%%%%%%%%%%%%%%%%%%%%%%%%%%%%%%%%

\section {Week \theweeknum \addtocounter{weeknum}{1} ?}

\subsubsection *{Dates} Mar 30 to Apr 3

\subsubsection *{Lecture \thelectnum \addtocounter{lectnum}{1} ?}

?

\subsubsection *{Lecture \thelectnum \addtocounter{lectnum}{1} ?}

?

\subsubsection *{Tutorial \thetutnum \addtocounter{tutnum}{1} ?}

NA

\subsubsection *{Textbook Reading (Ghezzi, H\&S or other)}

?

\subsubsection *{Assignment Progress}

?

\subsubsection *{Midterm/Final Review Progress}

?

\subsubsection *{Reflection Relating Course Topics, Other Courses, Other Experiences}

?

%%%%%%%%%%%%%%%%%%%%%%%%%%%%%%%%%%%%%%%%%%%%%%%%%%%%%%%%%%%%%%%%%%%%%%%%%%

\section {Week \theweeknum \addtocounter{weeknum}{1} ?}

\subsubsection *{Dates} Apr 6 to Apr 7

\subsubsection *{Lecture \thelectnum \addtocounter{lectnum}{1} ?}

?

\subsubsection *{Tutorial \thetutnum \addtocounter{tutnum}{1} ?}

NA

\subsubsection *{Textbook Reading (Ghezzi, H\&S or other)}

?

\subsubsection *{Assignment Progress}

?

\subsubsection *{Midterm/Final Review Progress}

?

\subsubsection *{Reflection Relating Course Topics, Other Courses, Other Experiences}

?

%%%%%%%%%%%%%%%%%%%%%%%%%%%%%%%%%%%%%%%%%%%%%%%%%%%%%%%%%%%%%%%%%%%%%%%%%%

\end{document}